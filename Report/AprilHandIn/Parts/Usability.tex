\chapter{Design choices}
\label{Design choices}
In this chapter we discuss our decisions regarding testing and design of our user interface.

\section{Usability testing}
\label{Design choices_Usability testing}
When the time came to start working on the client part of the project, we where hesitant with starting to code right away since all members of the group knew programs where a bad UI had caused it to flunk. We therefore dicided to take a more planned appraoch on how to design our UI, this envolv making usability testing, the aim of making usability tests is to observe people usering our product to discover errors and areas that can be improved,

\subsection{How to make a usability test}
\label{Design choices_How to make a usability test}
To be able to make a usability test in the first place we need a interface for the user to test, so we sat down as a team and discused how our clients design should look and function, when then made a couple of paper mockups for our user to test in the first usability test.
Before we started the usability test we had prepared some usability goals, which if meet would convince us that the current UI design should be our finale UI design.

List of usability goals:
\begin{itemize}
\item The user should be able to finish all given tasks within a time periode of 45 sec.
\item The user should be able to maneuver around the client without need to ask the tester qustions
\item The user should be positiv of the design.
\item The user should be able to recommend the service to his/her friends.
\end {itemize}

\subsection{Usability scenarios}
\label{Design choices_Usability scenarios}
For the usability test itself we had created a couple of scenarios, these scenarios was designed so that the user would be forced to navigate through all the clients functionalities.

List of usability scenarios:
\begin{itemize}
\item Your have heard of this new movie rental service and you would like to sign up for it.
\item You would like to rent batman the begining from the service.
\item You would like to see what movies that are most popoular at the moment.
\item You have gotten a new email and would like to change your profile so it use your new email.
\item As a movie company employee you would like to upload some of you're companys movies to the service.
\item You have uploaded a movie but it's with the wrong title change it
\item As a admin for the service you've seen some companys upload explicit material to service, delete those companys from the service.
\item  You would like to see a list of all the users who are using the service.
\end {itemize}

\subsection{Areas to improve}
\label{Design choices_Are to improve}
After the first round of test we found that the user had many difficulties with navigating through the program, specifically when ask to do the scenarios which involved saving and editing of information, when the user was asked why they fund those task so diffcult and the answer was that the felt a lack of confirmation on their action.
This came as a surprise since we had made  mockups of a client which we found was clear of redundent data and functionalitys, but apparently the need for confirmation is so ingrained in to days average user that it can't be removed with out the user missing it.
After the first test we then went about to solving the problems that had occurred, these was solved by adding more save buttons and some pop-up notification boxses asking for the usual "Are you sure you want to edit the information", when this was done we set on to make the alfa version of the client so that we could make a finally usability test with a digital mock up.










