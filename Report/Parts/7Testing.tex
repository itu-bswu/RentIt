\chapter{Testing}
\label{Testing}

\section{Strategy}
\label{Testing_Strategy}

\subsection{Test types}
\label{Testing_Strategy_Types}

\subsubsection{Scenario-level tests}
\label{Testing_Strategy_Types_Scenario}

\subsubsection{Service-level tests}
\label{Testing_Strategy_Types_Service}

\subsubsection{Graphical user interface tests}
\label{Testing_Strategy_Types_EndUser}
Testing that the service works is not the only important thing to do. It is also important to make sure that the graphical user interface works, because otherwise the user will not be able to use the service for anything. We split the testing of the graphical user interface into two parts: Automatic tests, and manual tests. These will be further described below.

\paragraph{Automatic GUI tests}
The automatic gui tests have been made through the use of the Coded UI Tests in Visual Studio 2010. When one creates such a test, one can record all the actions that are taken, and save these. When a test has been recorded, it can be run any amount of times, and it will take the same actions every time. Using this, we made automatic tests for all the basic features, such as signing in, logging in, searching for movies, etc.

\paragraph{Manual GUI test}
While we could automate some tests, there were others that were not worth automating. Automating things like upload/download would be too hard, as file directories change from computer to computer. This meant that we had to test certain functionalities manually, simply by opening the client and going through the necessary steps.

\subsection{Regression tests}
\label{Testing_Strategy_Regression}

\subsection{Usability tests}
\label{Testing_Strategy_Usability}
When designing a user interface you have to take into account that not all users is equally proficient in navigating IT systems, therefore we have to design a interface which is easy to use. To accomplish this we did a couple of usability tests. Usability tests, is a testing technique which focuses on the usability of a user interface, this is measured in non-functional requirements. For usability testing you need a mockup to test against, you then make a list of usability goals\footnote{See appendix for our usability goals} if these goals is fulfilled then you have the user interface that you wanted. For the usability test itself you make a list of scenarios that your user shall go through\footnote{See appendix for our usability scenarios}, while performing the scenarios, the user is asked to think aloud, such that the overseer of the test can take notes on how to improve the system.

The way we went about doing our usability tests, was to first set down as a team and create some paper mockups, which we found user friendly and had high ease-of-use. We then made some usability goals which if fulfilled, would ensure us that our interfaced was indeed user friendly and had a high ease-of-use. With these we made our first usability test on the paper mockups, we then got the feedback that even though our interface was easy to navigate through, we lacked user conformation all our test users felt uncertain that their actions was saved in the database.

For the second usability test we created a digital version of our paper mockups, but this time we added some dialog and confirmationboxses to ensure that the user didn't feel that their changes would go unsaved. Besides that we change abit of the design but without deviating to much from the paper mockups. We then went about doing the second usability test with the same goal and scenarios as before, this time we got no comment on the missing of confirmation in the program, but we did get the feedback that some of our paths was to obscured and not very intuitively. Generally the feedback told us that we needed to do something about our navigation and of our and how it was prestended to the user.


\section{Test results}
\label{Testing_Results}

\subsection{Code coverage}
\label{Testing_Results_Coverage}

\subsection{Usability results}
\label{Testing_Results_Usability}

\subsection{Results of GUI workflows}
\label{Testing_Results_Workflows}

\section{Reflection on test strategy}
\label{Testing_Reflection}

\subsection{Ideas for improvement}
\label{Testing_Reflection_improvements}
