\section{F\# Handins - Jacob Claudius Grooss}
\label{Appendix_FSharp_Grooss}

\subsection{HandIn 1}
\label{Appendix_FSharp_Grooss_1}
\begin{lstlisting}
module HandIn1
//Exercise 1
let sqr x = x * x

//Exercise 2
let pow x n = System.Math.Pow(x, n)

//Exercise 3
let dup (s:string) = s + s

//Exercise 4
let rec dupn (s:string, x) = if x = 0 then s else s + dupn(s, x - 1)

//Exercise 5
let timediff (hh1, mm1)(hh2, mm2) = (hh2 * 60 + mm2) - (hh1 * 60 + mm1)

//Exercise 6
let minutes (hh, mm) = timediff (00,00)(hh,mm)
\end{lstlisting}
\subsection{HandIn 2}
\label{Appendix_FSharp_Grooss_2}
\begin{lstlisting}
//Exercise 7
let rec downTo n = if n < 1 then [] else n :: downTo(n - 1)

let rec downTo2 n = 
    match n with
		| n when n < 1 -> []
        | 1 -> [1]
        | _ -> n :: downTo(n-1)

//Exercise 8
let rec removeEven (xs: int list) =
    match xs with
        | [] -> []
        | [xs] -> [xs]
        | xs :: xy :: rs -> xs :: removeEven(rs)

//Exercise 9
let rec combinePair (xs: int list) =
    match xs with
        | [] -> []
        | [xs] -> []
        | xs :: xy :: rs -> xs :: combinePair(rs)

//Exercise 10
let explode (s:string) =
    s.ToCharArray() |> List.ofArray

let rec explode2 (s:string) =
    match s with
        | s when s.Length < 1 -> []
        | _ -> s.[0] :: explode2 (s.Substring 1)

//Exercise 11
let implode (s:char list) =
    List.foldBack (fun str ch -> string(str) + string(ch)) s ""
    
let implodeRev (s:char list) =
    List.fold (fun str ch -> string(ch) + string(str)) "" s

//Exercise 12
let toUpper (s:string) =
    implode (List.map (fun x -> System.Char.ToUpper x) (explode s))

let toUpper1 (s:string) =
    explode >> (List.map (System.Char.ToUpper)) >> implode

let toUpper2 (s:string) =
    explode s |> (implode << List.map System.Char.ToUpper)

//Exercise 13
let palindrome (s:string) =
    (explode s |> implodeRev |> toUpper) = toUpper s

//Exercise 14
let rec ack (m, n) =
    match (m, n) with
        | (m, n) when m < 0 || n < 0 -> failwith "The Ackermann function 
				is defined for non negative numbers only."
        | (m, n) when m = 0 -> n + 1
        | (m, n) when n = 0 -> ack (m - 1, 1)
        | (m, n) -> ack (m-1, ack(m, n-1))

//Exercise 15
let time f =
    let start = System.DateTime.Now in
    let res = f () in
    let finish = System.DateTime.Now in
    (res, finish - start);

let timeArg1 f a = time(fun () -> f(a))
\end{lstlisting}
\subsection{HandIn 3}
\label{Appendix_FSharp_Grooss_3}
\begin{lstlisting}
type 'a BinTree =
    | Node of 'a * 'a BinTree * 'a BinTree
    | Leaf;;

let intBinTree = Node(43, Node(25, Node(56, Leaf, Leaf), Leaf),
Node(562, Leaf, Node(78, Leaf, Leaf)));;

//Exercise 16
let rec inOrder tree =
    match tree with
    | Leaf -> []
    | Node (n, treeL, treeR) ->
    inOrder treeL @ n :: inOrder treeR;;

//Exercise 17
let rec mapInOrder (funct: 'a -> 'b) tree =
    match tree with
    | Leaf -> Leaf
    | Node (n, treeL, treeR) ->
        let left = mapInOrder funct treeL
        let value = funct n
        let right = mapInOrder funct treeR
        Node (value, left, right);;

(* They traverse the tree in different orders,
which can give different results. *)

//Exercise 18
//Doesn't have the right signature, but this was the closest I could get
//to it while getting the correct result
let rec foldInOrder funct acc tree =
    match tree with
    | Leaf -> acc
    | Node (root, treeL, treeR) -> 
    funct (foldInOrder funct acc treeL) root (foldInOrder funct acc treeR);;

let func left root right = left + root + right;;

let seed = 1;;

let testingFol = foldInOrder func seed intBinTree;;

//Exercise 19 / 21 / 22
type expr =
    | Const of int
    | If of expr * expr * expr
    | Bind of string * expr * expr
    | Var of string
    | Prim of string * expr * expr

let rec evaluate expr (dict: 
System.Collections.Generic.Dictionary<string, expr>) =
    match expr with
    | Const(i) -> 
        i
    | If(expr1, expr2, expr3) ->
        if ((evaluate expr1 dict) > 0 || (evaluate expr1 dict) < 0) 
        then (evaluate expr2 dict) else (evaluate expr3 dict)
    | Bind(var, value, expr1) ->
        dict.Add(var, value)
        evaluate expr1 dict
    | Var(text) when dict.ContainsKey(text) ->
        evaluate (dict.Item text) dict
    | Var(text) -> 
        failwithf "Unknown variable '\%s'" text
    | Prim("-", expr1, expr2) -> 
        evaluate expr1 dict - evaluate expr2 dict
    | Prim("+", expr1, expr2) -> 
        evaluate expr1 dict + evaluate expr2 dict
    | Prim("max", expr1, expr2) -> 
        (List.max [evaluate expr1 dict; evaluate expr2 dict])
    | Prim("min", expr1, expr2) -> 
        (List.min [evaluate expr1 dict; evaluate expr2 dict])
    | Prim("=", expr1, expr2) -> 
        if (evaluate expr1 dict).Equals(evaluate expr2 dict) then 1 else 0
    | Prim(opr, _, _) -> 
        (printfn "Operation '\%s' not supported" opr; 0);;

let eval expr =
    evaluate expr (new System.Collections.Generic.Dictionary<string, expr>());;

//Exercise 20
let testingMinus = eval (Prim("-", Const 10, Const 5));;
let testingPlus = eval (Prim("+", Const 10, Const 5));;
let testingMx = eval (Prim("max", Const 10, Const 5));;
let testingMin = eval (Prim("min", Const 10, Const 5));;
let testingEquals = eval (Prim("=", Const 10, Const 5));;

let testingIf1 = eval (If(Const 3, Const 20, Const 18));;
let testingIf2 = eval (If(Const 0, Const 20, Const 18));;

//Exercise 23
let testingBindVal1 = eval (Bind("troll", Const 20, 
Bind("anti-troll", Const 42, Bind("super-troll", Var "troll", 
Var "anti-troll"))));;
let testingBindVal2 = eval (Bind("what", 
Bind("happens", Const 20, Var "happens"), Var "what"));;
let testingBindVal3 = eval (Bind("lol", Const 1337, Var "lol"));;
let testingVarFail1 = eval (Var("troll"));;
let testingVarFail2 = eval (Bind("fail", Const 117, Var "troll"));;
\end{lstlisting}

\subsection{HandIn 4 \& 5}
\label{Appendix_FSharp_Grooss_4and5}
\begin{lstlisting}
let findBot (card:card) = card.bot
let findTop (card:card) = card.top
let findLeft (card:card) = card.left
let findRight (card:card) = card.right

let pp_clothes clothes =
  match clothes with
    | RED_JACKET     -> "RED_JACKET    "
    | RED_TROUSERS   -> "RED_TROUSERS  "
    | GREEN_JACKET   -> "GREEN_JACKET  "
    | GREEN_TROUSERS -> "GREEN_TROUSERS"
    | BLUE_JACKET    -> "BLUE_JACKET   "
    | BLUE_TROUSERS  -> "BLUE_TROUSERS "
    | BROWN_JACKET   -> "BROWN_JACKET  "
    | BROWN_TROUSERS -> "BROWN_TROUSERS"

let rec splitNth (n, xs) = 
    match (n, xs) with
    | (0, xs) -> ([], xs)
    | (n, x :: xs) -> let (a, b) = splitNth (n - 1, xs);
                      (x :: a, b)
    | _ -> failwith "The input is out of range"

let rec pp_cards (cards : card list) = 
    match cards with
    | [] -> ()
    | card when cards.Length <= 4 -> pp_row (cards)
    | _ -> let (a, b) = splitNth(colNo, cards);
      pp_row a; pp_cards b

let matchClothes clothe1 clothe2 =
  match (clothe1,clothe2) with
    | (RED_JACKET,RED_TROUSERS) -> true
    | (RED_TROUSERS,RED_JACKET) -> true
    | (GREEN_JACKET,GREEN_TROUSERS) -> true
    | (GREEN_TROUSERS,GREEN_JACKET) -> true
    | (BLUE_JACKET,BLUE_TROUSERS) -> true
    | (BLUE_TROUSERS,BLUE_JACKET) -> true
    | (BROWN_JACKET,BROWN_TROUSERS) -> true
    | (BROWN_TROUSERS,BROWN_JACKET) -> true
    | _ -> false

let matchTop (row, col, cards) card =
  if row > 0 then
    let topCard = List.nth cards ((findIndexInList (row, col))-colNo)
    matchClothes (findBot topCard) (findTop card)
  else true

let matchLeft (row, col, cards) card =
  if col > 0 then
    let leftCard = List.nth cards ((findIndexInList(row, col))-1)
    matchClothes (findRight leftCard) (findLeft card)
   else true

let matchBot (row, col, cards) card = 
  if row > 0 then
    let botCard = List.nth cards ((findIndexInList (row, col))+colNo)
    matchClothes (findTop botCard) (findBot card)
  else true

let matchRight (row, col, cards) card =
  if col > 0 then
    let rightCard = List.nth cards ((findIndexInList(row, col))+1)
    matchClothes (findLeft rightCard) (findRight card)
   else true

(* There is ONE error in the code below - and it never terminates   *)
(* If you correct this one error - everything will work just fine   *)
(* You must explain the error as a comment here:                   
   The add function (line 256) was called with a 0 instead of a 1, 
   so it never changed the position of where the cards were matched. 
   This means that the cards constantly were matched on the first
   position (0, 0), where no other cards were around it.  This would
   always succeed, and therefore it kept piling the cards on top of
   each other, and therefore it never terminated.                   *)
let rec findSol rest alreadyTried ((row,col,cards) as board) sols =
  match (rest,alreadyTried) with
    | ([],[]) -> board::sols 
	(* No rest and alreadyTried is empty, that is, solution found *)
    | ([],_) -> sols         
	(* No solution if alreadyTried is non empty. *)
    | (x::rest, alreadyTried) ->
      let sols' = 
        if Match board x then
          (* If there is a match, then go on with the rest of the cards *)
          let (row', col') = add 1 (row, col)
          findSol(rest@alreadyTried) [] (row', col', cards@[x]) sols
        else sols (* If no match then no new solutions found. *)
      (* Put the card x in alreadyTried and move on. *)
      findSol rest (x::alreadyTried) board sols'

\end{lstlisting}