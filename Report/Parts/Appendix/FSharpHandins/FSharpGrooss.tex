\section{F\# Handins - Jacob Claudius Grooss}
\label{Appendix_FSharp_Grooss}

\subsection{HandIn 1}
\label{Appendix_FSharp_Grooss_1}
\begin{lstlisting}
module HandIn1
//Exercise 1
let sqr x = x * x

//Exercise 2
let pow x n = System.Math.Pow(x, n)

//Exercise 3
let dup (s:string) = s + s

//Exercise 4
let rec dupn (s:string, x) = if x = 0 then s else s + dupn(s, x - 1)

//Exercise 5
let timediff (hh1, mm1)(hh2, mm2) = (hh2 * 60 + mm2) - (hh1 * 60 + mm1)

//Exercise 6
let minutes (hh, mm) = timediff (00,00)(hh,mm)
\end{lstlisting}
\subsection{HandIn 2}
\label{Appendix_FSharp_Grooss_2}
\begin{lstlisting}
//Exercise 7
let rec downTo n = if n < 1 then [] else n :: downTo(n - 1)

let rec downTo2 n = 
    match n with
		| n when n < 1 -> []
        | 1 -> [1]
        | _ -> n :: downTo(n-1)

//Exercise 8
let rec removeEven (xs: int list) =
    match xs with
        | [] -> []
        | [xs] -> [xs]
        | xs :: xy :: rs -> xs :: removeEven(rs)

//Exercise 9
let rec combinePair (xs: int list) =
    match xs with
        | [] -> []
        | [xs] -> []
        | xs :: xy :: rs -> xs :: combinePair(rs)

//Exercise 10
let explode (s:string) =
    s.ToCharArray() |> List.ofArray

let rec explode2 (s:string) =
    match s with
        | s when s.Length < 1 -> []
        | _ -> s.[0] :: explode2 (s.Substring 1)

//Exercise 11
let implode (s:char list) =
    List.foldBack (fun str ch -> string(str) + string(ch)) s ""
    
let implodeRev (s:char list) =
    List.fold (fun str ch -> string(ch) + string(str)) "" s

//Exercise 12
let toUpper (s:string) =
    implode (List.map (fun x -> System.Char.ToUpper x) (explode s))

let toUpper1 (s:string) =
    explode >> (List.map (System.Char.ToUpper)) >> implode

let toUpper2 (s:string) =
    explode s |> (implode << List.map System.Char.ToUpper)

//Exercise 13
let palindrome (s:string) =
    (explode s |> implodeRev |> toUpper) = toUpper s

//Exercise 14
let rec ack (m, n) =
    match (m, n) with
        | (m, n) when m < 0 || n < 0 -> failwith "The Ackermann function 
				is defined for non negative numbers only."
        | (m, n) when m = 0 -> n + 1
        | (m, n) when n = 0 -> ack (m - 1, 1)
        | (m, n) -> ack (m-1, ack(m, n-1))

//Exercise 15
let time f =
    let start = System.DateTime.Now in
    let res = f () in
    let finish = System.DateTime.Now in
    (res, finish - start);

let timeArg1 f a = time(fun () -> f(a))
\end{lstlisting}
\subsection{HandIn 3}
\label{Appendix_FSharp_Grooss_3}
\begin{lstlisting}
\end{lstlisting}
\subsection{HandIn 4 \& 5}
\label{Appendix_FSharp_Grooss_4and5}
\begin{lstlisting}
\end{lstlisting}