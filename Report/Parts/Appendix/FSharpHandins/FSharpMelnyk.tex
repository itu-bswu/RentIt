\section{F\# Handins - Jakob Melnyk}
\label{Appendix_FSharp_Melnyk}

\subsection{HandIn 1}
\label{Appendix_FSharp_Melnyk_1}
\begin{lstlisting}
module Module1

// Exercise 1
let sqr x = x*x

// Exercise 2
let pow x n = System.Math.Pow(x, n)

// Exercise 3
let dup s : string = s + s

// Exercise 4
let rec dupn (s:string) x = 
	if x>=1 then (if x = 1 then s else s + dupn s (x-1)) else ""

// Exercise 5
let timediff (hh1, mm1)(hh2, mm2) = (hh2*60 + mm2)-(hh1*60 + mm1)

// Exercise 6
let minutes (hh, mm) = timediff(00, 00)(hh, mm)
\end{lstlisting}
\subsection{HandIn 2}
\label{Appendix_FSharp_Melnyk_2}
\begin{lstlisting}
module Module2

// Exercise 7
let rec downTo x = 
	if x < 1 then [] else (if x = 1 then [x] else x :: downTo (x - 1))

let rec downTo2 x = 
    match x with
    | x when x < 1 -> []
    | 1 -> [1]
    | _ -> x :: downTo2 (x - 1)

// Exercise 8
let rec removeEven (x:int list) = 
    match x with 
    | [] -> []
    | [xs] -> [xs]
    | xs :: ys :: zs -> xs :: removeEven zs

// Exercise 9
let rec combinePair (x:int list) : (int * int) list = 
    match x with
    | [] -> []
    | [xs] -> []
    | xs :: ys :: zs -> (xs, ys) :: combinePair zs

// Exercise 10
let explode (s:string) = List.ofArray (s.ToCharArray())

let rec explode2 (s:string) : char list = 
    match s with
    | s when s.Length < 1  -> []
    | _ -> s.[0] :: explode2 (s.Substring 1)

// Exercise 11
let implode (cl:char list) : string = 
		List.foldBack (fun elem acc -> string(elem) + string(acc) ) cl ""

let implodeRev (cl:char list) : string = 
		List.fold (fun elem acc -> string(acc) + string(elem) ) "" cl 

// Exercise 12
let toUpper (s:string) = implode (List.map System.Char.ToUpper (explode s))

let toUpper1 = explode >> List.map System.Char.ToUpper >> implode

let toUpper2 (s:string) = explode s |> (implode << List.map System.Char.ToUpper)

// Exercise 13
let palindrome (s:string) = (explode s |> implodeRev |> toUpper) = toUpper s

// Exercise 14
let rec ack (m, n) = 
    match (m, n) with
    | (m, n) when m < 0 || n < 0 -> failwith "The Ackermann function 
				is defined for non negative numbers only."
    | (m, n) when m = 0 -> n + 1
    | (m, n) when n = 0 -> ack (m - 1, 1)
    | (m, n) -> ack(m - 1, ack (m, n - 1))

// Exercise 15
let time f = 
    let start = System.DateTime.Now in
    let res = f () in
    let finish = System.DateTime.Now in
    (res, finish - start)

let timeArg1 f a = time(fun () -> f(a))
\end{lstlisting}
\subsection{HandIn 3}
\label{Appendix_FSharp_Melnyk_3}
\begin{lstlisting}
odule FSharpHandIn3

type 'a BinTree =
    Leaf
    |   Node of 'a * 'a BinTree * 'a BinTree

let intBinTree = 
    Node(
        43, 
        Node(25, Node(56,Leaf, Leaf), Leaf), 
        Node(562, Leaf, Node(78, Leaf, Leaf))
        )

// Exercise 16
let rec inOrder tree = 
    match tree with
    Leaf -> []
    |   Node(n, treeL, treeR) -> inOrder treeL @ [n] @ inOrder treeR

// Exercise 17
let rec mapInOrder (f:'a -> 'b) (tree:'a BinTree) : 'b BinTree = 
    match tree with
    Leaf -> Leaf
    |   Node(n, treeL, treeR) -> 
            let left = mapInOrder f treeL
            let root = f(n)
            let right = mapInOrder f treeR
            Node(root, left, right)
(*Example: 
The result tree should always be the same, as the function should 
access all the elements no matter what. 
The reason the individual nodes may not contain the same information 
could be that the function depends on the order in which 
the elements are accessed.*)

// Exercise 18
let rec foldInOrder f a t = 
    match t with
    | Leaf -> a
    | Node(x, leftTree, rightTree) ->
    let left = foldInOrder f a leftTree
    foldInOrder f (f x left) rightTree

// Exercise 19 & 21 & 22
type expr =
    | Const of int
    | If of expr * expr * expr
    | Bind of string * expr * expr
    | Var of string
    | Prim of string * expr * expr

let rec evalN expr (d:System.Collections.Generic.Dictionary<string, expr>) =
    match expr with
    | Const i -> i
    | Prim("-", expr1, expr2) -> 
        evalN expr1 d - evalN expr2 d
    | Prim("+", expr1, expr2) -> 
        evalN expr1 d + evalN expr2 d
    | Prim("max", expr1, expr2) -> 
        List.max [evalN expr1 d; evalN expr2 d]
    | Prim("min", expr1, expr2) -> 
        List.min [evalN expr1 d; evalN expr2 d]
    | Prim("=", expr1, expr2) -> 
        if evalN expr1 d = evalN expr2 d then 1 else 0
    | If(expr1, expr2, expr3) -> 
        if evalN expr1 d <> 0 then evalN expr2 d else evalN expr3 d
    | Bind(var, value, expr1) -> 
        d.Add(var, value) 
        evalN expr1 d
    | Var(name) when d.ContainsKey(name) -> 
        evalN (d.[name]) d
    | Var(name) -> 
        failwithf "Unknown variable '%s'" name
    | Prim(opr, _, _) -> 
        (printfn "Operation %s not supported" opr; 0)

let eval expr = 
    evalN expr (new System.Collections.Generic.Dictionary<string, expr>())

// Exercise 20
let testMinus = 
    eval (Prim("-",Const(20),Const(30))) // Expected result = -10
let testPlus = 
    eval (Prim("+",Const(20),Const(30))) // Expected result = 50
let testMax = 
    eval (Prim("max",Const(20),Const(30))) // Expected result = 30
let testMin = 
    eval (Prim("min",Const(20),Const(30))) // Expected result = 20
let testEqualFalse = 
    eval (Prim("=",Const(20),Const(30))) // Expected result = 0
let testEqualTrue = 
    eval (Prim("=",Const(20),Const(20))) // Expected result = 1

// Exercise 23
let testBindOne = // Expected result = 57
    eval (Bind("p", Prim("+", Const(13), Const(29)), Prim("+", Var("p"), Const(15)))) 
let testBindTwo = // Expected result = -16
    eval (Bind("x", Prim("-", Const(13), Const(29)), Prim("+", Var("x"), Const(15)))) 
let testBindThree = // Expected result = 97
    eval (Bind("x", Const(97), Bind("y", Const(3), Prim("max", Var("x"), Var("y"))))) 
let testBindFour = // Expected result = 0
    eval (Bind("x", Const(97), Bind("y", Const(3), Prim("=", Var("x"), Var("y"))))) 
let testBindFive =  // Fail case
    eval (Bind("x", Prim("+", Const(13), Const(29)), Prim("+", Var("y"), Const(15))))
\end{lstlisting}
\subsection{HandIn 4 \& 5}
\label{Appendix_FSharp_Melnyk_4and5}
\begin{lstlisting}
\end{lstlisting}