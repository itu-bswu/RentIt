\section{F\# Handins - Niklas Hansen}
\label{Appendix_FSharp_Niklas}

\subsection{HandIn 1}
\label{Appendix_FSharp_Niklas_1}
\begin{lstlisting}
// Author: Niklas Hansen <nikl@itu.dk>
module Handin1

// Exercise 1
let sqr (x:int) =
    x * x

// Exercise 2
let pow (x:float) (y:float) =
    x ** y

// Exercise 3
let dup (s:string) =
    s + s

// Exercise 4 - v1
let rec dupn (s:string) (n:int) =
    match n with
    | 0 -> ""
    | _ -> s + dupn s (n-1)

// Exercise 4 - v2
//let rec dupn (s:string) = function
//    | 0 -> ""
//    | n -> s + dupn s (n-1)

// Exercise 5
let timediff (h1:int, m1:int) (h2:int, m2:int) = 
    ((h2 * 60) + m2) - ((h1 * 60) + m1)

// Exercise 6
let minutes (hh:int, mm:int) =
    timediff (00, 00) (hh, mm)


printfn "1. Sqr 3: %i" (sqr 3)
printfn "2. pow 3 2: %f" (pow 3.0 2.0)
printfn "3: dup \"Hi \": %s" (dup "Hi ")
printfn "4. dupn \"Hi \" 3: %s" (dupn "Hi " 3)

printfn "5a. timediff (12, 34) (11, 35): %i" (timediff (12, 34) (11, 35))
printfn "5b. timediff (12, 34) (13, 35): %i" (timediff (12, 34) (13, 35))

printfn "6a. minutes (14, 24): %i" (minutes (14, 24))
printfn "6b. minutes (23, 1):  %i" (minutes (23, 1))
\end{lstlisting}
\subsection{HandIn 2}
\label{Appendix_FSharp_Niklas_2}
\begin{lstlisting}
// Author: Niklas Hansen <nikl@itu.dk>
module Handin2

// Exercise 7a
let rec downTo (n:int) =
    if n > 0
        then n :: downTo (n-1)
        else []

// Exercise 7b
let rec downTo2 (n:int) =
    match n with
    | n when n <= 0 -> []
    | _ -> n :: downTo2 (n-1)

// Exercise 7b v2
//let rec downTo2 = function
//    | n when n <= 0 -> []
//    | n -> n :: downTo2 (n-1)

// Exercise 8
let rec removeEven = function
    | [] -> []
    | [n] -> [n]
    | n :: m :: tl -> n :: removeEven tl

// Exercise 9
let rec combinePair = function
    | []-> []
    | [n] -> []
    | n :: m :: tl -> (n, m) :: combinePair tl

// Exercise 10a
let explode (s:string) =
    let chars = s.ToCharArray()
    List.ofArray(chars)

// Exercise 10b
let rec explode2 (s:string) =
    match s with
    | "" -> []
    | _ -> s.Chars 0 :: explode2 (s.Remove(0, 1))

// Exercise 11a
let implode (c:char list) =
    List.foldBack (fun x y -> sprintf "%c%s" x y) c ""

// Exercise 11b
let implodeRev (c:char list) =
    List.fold (fun x y -> sprintf "%c%s" y x) "" c

// Exercise 12a
let toUpper (s:string) =
    implode (List.map (fun x -> System.Char.ToUpper(x)) (explode s))

// Exercise 12b
let toUpper1 (s:string) =
    (explode >> List.map (fun x -> System.Char.ToUpper(x)) >> implode) s

// Exercise 12c
let toUpper2 (s:string) =
    s |> (implode << List.map (fun x -> System.Char.ToUpper(x)) << explode)

// Exercise 13
let palindrome (s:string) =
    let org = s.ToLower().Replace(" ", "")
    let rev = new string (Array.rev (org.ToCharArray()))
    org = rev

// Exercise 14
let rec ack (m:int, n:int) =
    match m, n with
    | (0, n) -> n + 1
    | (m, 0) when m > 0 -> ack(m - 1, 1)
    | (m, n) when m > 0 && n > 0 -> ack(m - 1, ack(m, n - 1))
    | (m, n) -> failwith "Invalid input!"

// Addon for Exercise 15
let time f =
    let start = System.DateTime.Now
    let res = f ()
    let finish = System.DateTime.Now
    (res, finish - start)

// Exercise 15
let timeArg1 f a =
    time (fun () -> f a)


printfn "7a. downTo 5: %s" ((downTo 5).ToString())
printfn "7a. downTo -3: %s" ((downTo -3).ToString())
printfn "7b. downTo2 5: %s" ((downTo2 5).ToString())

printfn "8. removeEven [1; 2; 3; 4; 5]: %s" ((removeEven [1; 2; 3; 4; 5]).ToString())
printfn "8. removeEven []: %s" ((removeEven []).ToString())
printfn "8. removeEven [1]: %s" ((removeEven [1]).ToString())

printfn "9. combinePair [1; 2; 3; 4]: %s" ((combinePair [1; 2; 3; 4]).ToString())
printfn "9. combinePair [1; 2; 3]: %s" ((combinePair [1; 2; 3]).ToString())
printfn "9. combinePair [1; 2]: %s" ((combinePair [1; 2]).ToString())
printfn "9. combinePair []: %s" ((combinePair []).ToString())
printfn "9. combinePair [1]: %s" ((combinePair [1]).ToString())

printfn "10a. explode \"star\": %s" ((explode "star").ToString())
printfn "10b. explode2 \"star\": %s" ((explode2 "star").ToString())

printfn "11a. implode ['a';'b';'c']: %s" (implode ['a';'b';'c'])
printfn "11b. implodeRev ['a';'b';'c']: %s" (implodeRev ['a';'b';'c'])

printfn "12a. toUpper \"Hej\": %s " (toUpper "Hej")
printfn "12b. toUpper1 \"Hej\": %s " (toUpper1 "Hej")
printfn "12c. toUpper2 \"Hej\": %s " (toUpper2 "Hej")

printfn "13. palindrome \"Anna\": %s" ((palindrome "Anna").ToString())
printfn "13. palindrome \"Ann\": %s" ((palindrome "Ann").ToString())

printfn "14. ack(3, 11): %i" (ack(3, 11))

printfn "Extra. time: %s" ((time (fun () -> ack (3,11))).ToString())
printfn "15. timeArg1 ack (3, 11): %s" ((timeArg1 ack (3, 11)).ToString())

System.Console.ReadKey(true)
\end{lstlisting}
\subsection{HandIn 3}
\label{Appendix_FSharp_Niklas_3}
\begin{lstlisting}
// Author: Niklas Hansen <nikl@itu.dk>
module Handin3
open System.Collections.Generic

// Prerequisites
type 'a BinTree = 
    Leaf
  | Node of 'a * 'a BinTree * 'a BinTree

// Test data
let tree = Node(43, Node(25, Node(56, Leaf, Leaf), Leaf), 
					Node(562, Leaf, Node(78, Leaf, Leaf)))

// Exercise 16
let rec inOrder (tree:'a BinTree) =
    match tree with
    | Leaf -> []
    | Node(elem, treeLeft, treeRight) -> inOrder treeLeft @ elem :: inOrder treeRight

// Exercise 17
// If the map function is doing exactly the same thing to all elements,
// then mapPostOrder will be different from mapInOrder, because of the 
// order of which are being processed at what time.
// Example: Let's say that the map function adds 1 to the first element,
// 2 to the next, 3 to the third element and so forth. Then it is very 
// important in which order the elements are being processed.
let rec mapInOrder (func:'a -> 'b) (tree:'a BinTree) =
    match tree with
    | Leaf -> Leaf
    | Node(elem, treeLeft, treeRight) ->
        let left = mapInOrder func treeLeft
        let node = func elem
        let right = mapInOrder func treeRight
        Node(node, left, right)

// Exercise 18
let rec foldInOrder f a tree =
    match tree with
    | Leaf -> a
    | Node(n, ltree, rtree) -> 
        let a' = foldInOrder f a ltree in
                 foldInOrder f (f n a') rtree

// Alternative (produces: ('a -> 'b -> 'b) -> 'b -> 'a BinTree -> 'b):
//let foldInOrder func acc input = List.foldBack func (inOrder input) acc
// Note: While both of these produces a list in order, this last one 
// iterates of the list from the back. In most cases this won't matter, 
// but it might, for the same reason given in exercise 17. But this last 
// approach produces the signature given in the assignment.

// Exercise 19-22
type expr =
    | Const of int
    | If of expr * expr * expr
    | Bind of string * expr * expr
    | Var of string
    | Prim of string * expr * expr

let rec evaluate expr (dict:Dictionary<string, expr>) =
    match expr with
        | Const i -> i
        | Prim(opr, expr1, expr2) ->
            let val1 = evaluate expr1 dict;
            let val2 = evaluate expr2 dict;
            match opr with
            | "-" -> val1 - val2
            | "+" -> val1 + val2
            | "=" -> if val1 = val2 then 1 else 0
            | "min" -> if val1 > val2 then val2 else val1
            | "max" -> if val2 > val1 then val2 else val1
            | _ -> failwithf "Operation %s not supported" opr
        | If(expr1, exprThen, exprElse) ->
            if evaluate expr1 dict <> 0
            then evaluate exprThen dict
            else evaluate exprElse dict
        | Bind(var, value, expr) ->
            dict.[var] <- value
            evaluate expr dict
        | Var(name) when dict.ContainsKey(name) -> evaluate (dict.Item name) dict
        | Var(name) -> failwithf "Undeclared variable '%s'" name

let eval expr =
    evaluate expr (new Dictionary<string, expr>())

// Exercise 20
printfn "20. 30-10: %i" (eval (Prim("-", Const 30, Const 10)))
printfn "20. 30+10: %i" (eval (Prim("+", Const 30, Const 10)))
printfn "20. 30=10: %i" (eval (Prim("=", Const 30, Const 10)))
printfn "20. min [30, 10]: %i" (eval (Prim("min", Const 30, Const 10)))
printfn "20. max [30, 10]: %i" (eval (Prim("max", Const 30, Const 10)))

// Exercise 23
printfn "23. #1: %i" (eval (Bind("x", Const 10,
                                Bind("y", Const 20, 
                                    Prim("+", Var "x", Var "y")))))
printfn "23. #2: %i" (eval (Bind("x", Prim("+", Const 123, Const 54),
                                Prim("max", Var "x", Const 150))))
printfn "23. #3: %i" (eval (Prim("=", Const 42, 
                                Bind("x", Const 24,
                                    Prim("+", Const 18, Var "x")))))
printfn "23. #4: %i" (eval (Bind("x", Const 42,
                                Bind("x", Const 43, 
                                    Var "x"))))
//printfn "23. #5: %i" (eval (Var "x")) // Expected failure
//printfn "23. #6: %i" (eval (Bind("y", Const 10, Var "x"))) // Expected failure


// Tests
printfn "16. inOrder tree: %s" ((inOrder tree).ToString())
printfn "17. mapInOrder (fun x -> x*2) tree: %s" 
				((mapInOrder (fun x -> x*2) tree |> inOrder).ToString())
printfn "18. foldInOrder: %s" ((foldInOrder (+) 0 tree).ToString())

// Prevent program exit
System.Console.ReadKey(true)
\end{lstlisting}
\subsection{HandIn 4 \& 5}
\label{Appendix_FSharp_Niklas_4and5}
\begin{lstlisting}
(* Define four functions that given a card with return one of the four clothes on the card. *)
let findBot (card:card) = card.bot
let findTop (card:card) = card.top
let findLeft (card:card) = card.left
let findRight (card:card) = card.right

(* Define a function that as a string returns a pretty print of the clothes. *)
let pp_clothes clothes =
  match clothes with
    | RED_JACKET     -> "RED_JACKET    "
    | RED_TROUSERS   -> "RED_TROUSERS  "
    | GREEN_JACKET   -> "GREEN_JACKET  "
    | GREEN_TROUSERS -> "GREEN_TROUSERS"
    | BLUE_JACKET    -> "BLUE_JACKET   "
    | BLUE_TROUSERS  -> "BLUE_TROUSERS "
    | BROWN_JACKET   -> "BROWN_JACKET  "
    | BROWN_TROUSERS -> "BROWN_TROUSERS"

(* Split list [x1,...,xN] in the lists [x1,...,xn-1] and [xn,...,xN] *)
(* where n >= 0 and n < N .                                          *)
(* Fx: splitNth (0,[1;2]) gives ([], [1; 2])                         *)
(*     splitNth (1,[1;2]) gives ([1], [2])                           *)
(*     splitNth (2,[1;2]) gives ([1; 2], [])                         *)
(*     splitNth (3,[1;2]) should die : 3 outside range of list       *)
(*     splitNth (-1,[1;2]) should die : -1 outside range of list     *)
let rec splitNth (n, xs) = 
    match (n, xs) with
    | (0, xs) -> ([], xs)
    | (n, x::xs) when n > 0 && n <= (xs.Length + 1) -> 
        let res = splitNth ((n-1), xs);
        (x :: (fst res), (snd res))
    | (n, _) -> die (sprintf "%i outside range of list" n)
(* (xs.Length + 1) must be used in the second rule, to conform to the 
    example input/outputs given as documentation above. *)

let rec pp_cards (cards:card list) =
    match cards with
    | [] -> ()
    | cards ->
        let split = if cards.Length < colNo then (cards, []) else splitNth (colNo, cards)
        pp_row (fst split)
        snd split |> pp_cards

let matchClothes clothe1 clothe2 =
  match (clothe1, clothe2) with
    | (RED_JACKET, RED_TROUSERS) -> true
    | (RED_TROUSERS, RED_JACKET) -> true
    | (GREEN_JACKET, GREEN_TROUSERS) -> true
    | (GREEN_TROUSERS, GREEN_JACKET) -> true
    | (BLUE_JACKET, BLUE_TROUSERS) -> true
    | (BLUE_TROUSERS, BLUE_JACKET) -> true
    | (BROWN_JACKET, BROWN_TROUSERS) -> true
    | (BROWN_TROUSERS, BROWN_JACKET) -> true
    | _ -> false

(* matchBot: Given a coordinate, match that card with the card immediately below. *)
(* Notice, cards at the bottom row fulfils this automatically.                    *)
let matchBot (row, col, cards) card = 
    if row < rowNo then
        let botCard = List.nth cards ((findIndexInList (row, col))-colNo)
        matchClothes (findTop botCard) (findBot card)
    else true

(* matchRight: Given a coordinate, match that card with the card immediately to the right. *)
(* Notice, cards at the rightmost column fulfils this automatically.                       *)
let matchRight (row, col, cards) card =
    if col < colNo then
        let rightCard = List.nth cards ((findIndexInList(row, col))-1)
        matchClothes (findLeft rightCard) (findRight card)
    else true

(* There is ONE error in the code below - and it never terminates *)
(* If you correct this one error - everything will work just fine *)
(* When trying to call findSol recursively, it must add 1 to the row and column.
     This was 0 in the original code, which means that the code never got any further. 
     Changing it to "add 1" fixed it.                   *)
let rec findSol rest alreadyTried ((row,col,cards) as board) sols =
  match (rest,alreadyTried) with
    | ([],[]) -> board::sols (* No rest and alreadyTried is empty, that is, solution found *)
    | ([],_) -> sols         (* No solution if alreadyTried is non empty. *)
    | (x::rest, alreadyTried) ->
      let sols' = 
        if Match board x then
          (* If there is a match, then go on with the rest of the cards *)
          let (row', col') = add 1 (row, col)
          findSol(rest@alreadyTried) [] (row', col', cards@[x]) sols
        else sols (* If no match then no new solutions found. *)
      (* Put the card x in alreadyTried and move on. *)
      findSol rest (x::alreadyTried) board sols'
\end{lstlisting}