\chapter{Manual}
\label{Manual}

\section{Client}
\label{Manual_Client}

\subsection{Navigating the client}
\label{Manual_Client_Navigation}
The following section will be a short manual on how to use the client when trying to perform two standard tasks: to rent a movie from the service, and to upload a moive to the service.

\subsubsection{Renting a movie}
\label{Manual_Client_Navigation_Rent}
The first thing that you see when opening the client, is the login page that asks for a username and a password. If you do not have an account, you can click the "Signup" button, which will direct you to another page. On this page, you can create a new account, after which you will be returned to the login page. When you log in, you will see a page with a list of all the movies that the database contain (see figure \ref{fig:Manual_Client_Navigation_Rent_List} on page \pageref{fig:Manual_Client_Navigation_Rent_List}).

\begin{figure}[h!] 
  \centering
 \includegraphics[width=0.5\textwidth]{Parts/Images/Manual/Listmovies}
\caption{List of movies on service}
\label{fig:Manual_Client_Navigation_Rent_List}
\end{figure}
 
A different page will be shown for the content provider, but that will be covered later.

To rent a movie, you'll first have to find the movie in the list. There are two ways of doing this: One way it to sort the list of movies after genre (if you know the genre of the movie). The other way is to do a search for the movie through the use of the search box by entering the name of the movie, and pressing the "Search" button.

If you use the search option, you will be directed to a new page, where you can see the results of the search (see figure \ref{fig:Manual_Client_Navigation_Rent_List} on page \pageref{fig:Manual_Client_Navigation_Rent_View}). This page looks like the previous page, but contains only the movies that matches the serch criteria(s). In this example, "Batman Begins" is the target, so it will be shown in this list. From this list, you can chose the movie ("Batman Begins"), and click the "View movie" button. This will lead you to a page that contains information about the movie in question (see figure \ref{fig:Manual_Client_Navigation_Rent_View} on page \pageref{fig:Manual_Client_Navigation_Rent_View}).

\begin{figure}[h!]  
  \centering
\includegraphics[width=0.5\textwidth]{Parts/Images/Manual/Viewmovie}
\caption{View movie}
\label{fig:Manual_Client_Navigation_Rent_View}
\end{figure}

From this screen, you chose the edition that you want to rent. In this example it will be the HD 1080P edition of "Batman Begins". When you have selected the wanted edition, you click the "Select edition" button, which directs you to a new page. This page is similar to the previous one, except for the fact that the list of editions is gone, and the "Select edition" button has been replaced by a "Download movie" button. Pressing the "Download movie" button will make a popup window appear, asking for the directory to save the movie to (see figure \ref{fig:Manual_Client_Navigation_Rent_Download} on page \pageref{fig:Manual_Client_Navigation_Rent_Download}).

\begin{figure}[h!]  
  \centering
\includegraphics[width=0.5\textwidth]{Parts/Images/Manual/Downloadmovie}
\caption{Download movie}
\label{fig:Manual_Client_Navigation_Rent_Download}
\end{figure}
 
When the movie has finished downloading, you can press the "Logout" button to log out of the service, and return to the login page. You can also close the program by pressing the red X at the top right of the screen, which will log you out and close the application completely.

\subsubsection{Uploading a movie}
\label{Manual_Client_Navigation_Upload}

If you log in as a content provider, you will be shown a different page than the normal user will see. On this page you will see a list of all your movies (see figure
\ref{fig:Manual_Client_Navigation_Upload_List} on page \pageref{fig:Manual_Client_Navigation_Upload_List}).

\begin{figure}[h!]  
  \centering
\includegraphics[width=0.5\textwidth]{Parts/Images/Manual/CPmovielist}
\caption{List of uploaded movies}
\label{fig:Manual_Client_Navigation_Upload_List}
\end{figure}
 
From this page, you can chose to register a new movie. To do this, you press the "Register movie" button, which will direct you to a new page (see figure \ref{fig:Manual_Client_Navigation_Upload_Register} on page \pageref{fig:Manual_Client_Navigation_Upload_Register}). On this page, you can give the movie a title, set the release date of the movie, write a description and chose the genres of the movie. 

When you have entered the information you want to, you can press the "Register movie" button to register the movie in the database. When you do this, a popup window will appear, and ask if you want to upload an edition to the registered movie right away (see figure \ref{fig:Manual_Client_Navigation_Upload_Popup} on page \pageref{fig:Manual_Client_Navigation_Upload_Popup}).

\begin{figure}[h!]  
  \centering
\includegraphics[width=0.5\textwidth]{Parts/Images/Manual/CPRegistermovie}
\caption{Register movie}
\label{fig:Manual_Client_Navigation_Upload_Register}
\end{figure}

\begin{figure}[h!]  
  \centering
\includegraphics[width=0.5\textwidth]{Parts/Images/Manual/RegisterPopup}
\caption{Upload edition popup box}
\label{fig:Manual_Client_Navigation_Upload_Popup}
\end{figure}

If you press "Yes", you will be directed to a page that allows you to upload editions (see figure \ref{fig:Manual_Client_Navigation_Upload_Edition} on page \pageref{fig:Manual_Client_Navigation_Upload_Edition}). On this page, you can write the title of the movie, the name of the edition and chose a file from your computer to upload. When the fields have been filled, press the "Upload edition" button, and the edition will be attached to the movie in the database, and will show up in the page that shows the movie information. When you have finished uploading the edition, you will be returned to your start page.

\begin{figure}[h!]  
  \centering
\includegraphics[width=0.5\textwidth]{Parts/Images/Manual/CPUploadEdition}
\caption{Upload edition}
\label{fig:Manual_Client_Navigation_Upload_Edition}
\end{figure}

\section{Service}
\label{Manual_Service}

\subsection{Using the API}
\label{Manual_Service_Usage}
The following section is a short manual on how to use the service API in custom client applications. It will follow the same two tasks as those in the client manual, only focusing on the service calls.

All methods in the API have the same format. The return type is always a boolean, which is true if the input was valid, but false if the input was invalid. All objects to be sent to the client is sent through either an out or ref parameter.

All methods are places in 4 different classes: \class{ContentBrowsing}, \class{UserManagement}, \class{RentalManagement}, and \class{ContentManagement}.

\subsection{Renting a movie}
\label{Manual_Service_Usage_Rent}
The first step to using the service is always either log in or sign up for a new user. The \method{Login} method in \class{UserManagement} takes a username and password, and sends back the logged in user object. The \method{SignUp} method takes a referenced \class{User} object, where at least username, password and email is filled out. The returned user object for both methods will contain a token property, which will be used in all subsequent method calls.

After login, there is several ways to get movies, all using the \class{ContentBrowsing} class. The \method{GetMovies} method can get all movies, newest movies, and most downloaded movies in either all or a specified genre. The genres can be retrieved with the \method{GetGenres} method. Alternatively, you can search for movie titles with the \method{Search} method.

To rent a movie, the \method{RentMovie} method in the \class{RentalManagement} class is used. This takes a movie edition. The list of editions for a movie can be retrieved with the \method{GetMovieInformation} method in the \class{ContentBrowsing} class. When a movie edition has been rented, it can be downloaded with the \method{DownloadFile} method.

\subsection{Uploading a movie}
\label{Manual_Service_Usage_Upload}
When you log in as a content provider, you get the ability to upload, edit and remove movies through the \class{ContentManagement} class. The \method{RegisterMovie} method registers a new movie in the system. It takes a referenced movie instance, which has to at least have a title. A release date can also be set, and if it's a day in the future, the movie won't be visible in the system before then. A movie can afterwards be edited and removed using the \method{EditMovie} and \method{DeleteMovie} method.

After a movie has been registered, versions of it can be uploaded with the \method{UploadEdition}, and afterwards be removed with the \method{DeleteEdition} method.
