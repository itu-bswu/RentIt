\section{Client}
\label{Design_Client}
This section covers the design decisions we made regarding the client implementation and Graphical User Interface (GUI) design.
\subsection{Analysis}
\label{Design_Client_Analysis}
We looked into what architectures were commonly used and suggested when developing a client with a \class{GUI} front-end and a \class{WCF}-service backend. We have seen (and used) the \class{Model-View-Control} (\class{MVC}\footnote{Wikipedia description of Model-View-Control\cite{WIKI-MVC}}) before this project, and we felt it fit nicely with having a model (the service), a view (\class{GUI}) and a controller to make it all work. 
\\While looking further into \class{MVC} and how we could apply to our client, we found the \class{Model-View-Viewmodel} (MVVM\footnote{References on Wikipedia \cite{WIKI-MVVM} and MSDN \cite{MSDN-WPF-MVVM}}) architecture pattern. The \class{MVVM} pattern is based largely on \class{MVC}, but is targeted at modern \class{UI} development platforms (such as HTML5, Windows Presentation Foundation and Silverlight).

Because we decided to use Windows Presentation Foundation (\class{WPF})\footnote{Decision described in section \ref{Implementation_Client} on page \pageref{Implementation_Client}.}, we decided that trying out the \class{MVVM}-pattern was a good idea. \class{MVVM} offers a complete seperation of the model (in our case the \class{WCF} service) and \class{GUI}. The viewmodels serve as translators (and sometimes logic functionality, depending on implementation).
\subsection[Architecture]{Our version of the MVVM Architecture}
\label{Design_Client_Architecture}
Having chosen the MVVM architecture, we decided to implement\footnote{How we implement this is described in section \ref{Implementation_Client_Architecture} on page \pageref{Implementation_Client_Architecture}.} our own version of it (instead of using frameworks set up to use MVVM). It gave us more control over what we wanted to do with the architecture, as well as letting us make any modifications we want to.

The biggest change we have made to the usual architecture\footnote{\class{View}(\class{GUI})-\class{Model}(database/datafiles)-\class{ViewModel}(translator).} is our interpretation of models. Instead of having models be the actual database, we use it as a communicator with the service. In the MSDN blog post on the MVVM architecture \cite{MSDN-WPF-MVVM} and how it can be used with WPF and a WCF service, the model is described as being the actual service. 

In our approach, we design the models as being seperate classes with an interface that the view models can use. This completely seperates the viewmodels from the service calls (meaning the model could actually be anything, as long as it implements the same interface). We will still use classes and objects from the service reference in the view models, but to change the data/service the model accesses, one could simply use the model to translate the types into the types from the service reference. This allows a modular approach to the system.

Because we have a modular approach to the model-viewmodel relationship, we feel the view-viewmodel relationship should be modular as well. Because of this, the implementation of the view models could change vastly without having any effect on the views (except if it changes interface).

\subsection{Graphical User Interface}
\label{Design_Client_GUI}
When we designed the GUI, we had two different approaches. One approach was to open a new window, every time the user was to access a new functionality. Another approach was to make the GUI change the page it showed, depending on what functionality the user was accessing. The design team tried both options and found that the having multiple windows open was too confusing.Since the second option felt more natrual and less clumsy, the design team chose the second option.

With the second option chosen, the design team had to begin creating the GUI. To keep some consistency in the GUI, and to allow the user to access most functionalities from anywhere, the top of the GUI had buttons that were shared between all screens added. Between the top and the bottom of the GUI, we 

With the one window design we had each functionallity page consist of three areas: In the top we had shared buttons for the user type, in the middel we represented the page content associated with the functionallity and in the bottom we had the buttons associated with the page content. See figure \ref{fig:Design_Client_GUI_Usability_design} on page \pageref{fig:Design_Client_GUI_Usability_design} for graphical representation.

\begin{figure}[h!]
  \centering
\includegraphics[width=0.60\textwidth]{Parts/Images/Design/GUIDesign}
\caption{Graphical representation of the overall GUI design}
\label{fig:Design_Client_GUI_Usability_design}
\end{figure}


\subsubsection{Usability}
\label{Design_Client_GUI_Usability}
When we began designing our GUI, we felt that it was important it was user friendly. We therefore choose to make usability tests (see \ref{Testing_Strategy_Usability}), since they always will grant some degree of usability if performed correctly. In total we conducted two usability tests.
\\The feedback from the first test told us that we didn't have enough user confirmation in the GUI. We discussed this in the design team and we came up with a solution which added dialog and confirmation boxses to a lot of our buttons which contained a save function (see figure \ref{fig:Design_Client_GUI_Usability_popup}on page \pageref{fig:Design_Client_GUI_Usability_popup}).

\begin{figure}[h!]
  \centering
\includegraphics[width=0.4\textwidth]{Parts/Images/Design/Confirmationbox}
\caption{Confirmation box}
\label{fig:Design_Client_GUI_Usability_popup}
\end{figure}

In the second usability test, we went from testing on a papermockup of our client, to testing on our client prototype, which incorporated the design changes from the first test. In this test, we received no feedback concerning the lack of conformation in the client, which meant that we had sovled the problem from the first test. We did however get feedback on the navigation of the client. Our test users found it hard to navigate to the correct pages during tests. They said that either they had to navigate through too many pages to get to the correct one or the buttons were named ambiguously. For example, one of the users thought that the view movie button would play the selected movie when clicked.

Unfortunately, the second usability test was carried out too late in the process, which meant that we didn't have time to incorporate changes to the client. If we had time, we would have done the following: We would have revised the naming of our buttons, such that there would be no confusion with regards to their functionality. In addition, we would makes changes to our "menu bar" at the top of the client, such that the user would allways have more pages to navigate to. This would make unnecessary navigation through other pages redundant.
