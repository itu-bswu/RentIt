\section{Client}
\label{Design_Client}
This section covers the design decisions we made regarding the client implementation and Graphical User Interface (GUI) design.
\subsection{Analysis}
\label{Design_Client_Analysis}
We looked into what architectures were commonly used and suggested when developing a client with a \class{GUI} front-end and a \class{WCF}-service backend. We have seen (and used) the \class{Model-View-Control} (\class{MVC}\footnote{Wikipedia description of Model-View-Control\cite{WIKI-MVC}}) before this project, and we felt it fit nicely with having a model (the service), a view (\class{GUI}) and a controller to make it all work. 
\\While looking further into \class{MVC} and how we could apply it to our client, we found the \class{Model-View-Viewmodel} (MVVM\footnote{References on Wikipedia \cite{WIKI-MVVM} and MSDN \cite{MSDN-WPF-MVVM}}) architecture pattern. The \class{MVVM} pattern is based largely on \class{MVC} but is targeted at modern \class{UI} development platforms (such as HTML5, Windows Presentation Foundation and Silverlight).

Because we decided to use Windows Presentation Foundation (\class{WPF})\footnote{Decision described in section \ref{Implementation_Client} on page \pageref{Implementation_Client}.}, we decided that trying out the \class{MVVM}-pattern was a good idea. \class{MVVM} offers a complete seperation of the model (in our case the \class{WCF} service) and \class{GUI}. The viewmodels serve as translators (and sometimes logic functionality, depending on implementation).
\subsection[Architecture]{Our version of the MVVM Architecture}
\label{Design_Client_Architecture}
Having chosen the MVVM architecture, we decided to implement\footnote{How we implement this is described in section \ref{Implementation_Client_Architecture} on page \pageref{Implementation_Client_Architecture}.} our own version of it (instead of using frameworks set up to use MVVM). It gave us more control over what we wanted to do with the architecture, as well as letting us make any modifications we wanted to.

The biggest change we have made to the usual version of the architecture\footnote{\class{View}(\class{GUI})-\class{Model}(database/datafiles)-\class{ViewModel}(translator).} is our interpretation of models. Instead of having models be the actual database, we use it as a communicator with the service. In the MSDN blog post on the MVVM architecture \cite{MSDN-WPF-MVVM} and how it can be used with WPF and a WCF service, the model is described as being the actual service. 

In our approach, we design the models as being seperate classes with an interface that the viewmodels can use. This completely seperates the viewmodels from the service calls (meaning the model could actually be anything, as long as it implements the same interface). We will still use classes and objects from the service reference in the viewmodels, but to change the data/service the model accesses, one could simply use the model to translate the types into the types from the service reference. This allows a modular approach to the system.

Because we have a modular approach to the model-viewmodel relationship, we feel the view-viewmodel relationship should be modular as well.  Due to that, the implementation of the viewmodels could change vastly without having any effect on the views (except if it changes interface).

\subsection{Graphical User Interface}
\label{Design_Client_GUI}
When we designed the GUI, we had two different approaches. One approach was to open a new window, every time the user was to access a new functionality. Another approach was to make the GUI change the page it showed, depending on what functionality the user was accessing. The design team tried both options and found that the having multiple windows open was too confusing.Since the second option felt more natrual and less clumsy, the design team chose the second option.

With the second option chosen, the design team had to begin creating the GUI. To keep some consistency in the GUI, and to allow the user to access most functionalities from anywhere, the top of the GUI had buttons that were shared between all screens added. Between the top and the bottom of the GUI, the page content associated with the current page was placed. At the bottom of the screen, buttons that were associated with the page content were added (see figure \ref{fig:Design_Client_GUI_Usability_design} on page \pageref{fig:Design_Client_GUI_Usability_design}).

\begin{figure}[h!]
  \centering
\includegraphics[width=0.60\textwidth]{Parts/Images/Design/GUIDesign}
\caption{Graphical representation of the overall GUI design}
\label{fig:Design_Client_GUI_Usability_design}
\end{figure}

\subsubsection{Usability}
\label{Design_Client_GUI_Usability}
When we began the design of the GUI, we decided that it was important that the GUI was user-friendly. Because of this, we chose to conduct some usability tests (see \ref{Testing_Strategy_Usability}). The reason for this, is that usability tests always will reveal flaws in the GUI, which allows the designers to take care of these flaws. 

The first usability test that we conducted was done with the use of a papermockup. The test told us that there was not enough user confirmation in the GUI. The test subjects were not sure that their changes had been saved, which caused some confusion among them. We took this feedback to heart, and added dialog- and confirmation boxes to the GUI, whereever changes were to be saved (see figure \ref{fig:Design_Client_GUI_Usability_popup}on page \pageref{fig:Design_Client_GUI_Usability_popup}).

\begin{figure}[h!]
  \centering
\includegraphics[width=0.4\textwidth]{Parts/Images/Design/Confirmationbox}
\caption{Confirmation box}
\label{fig:Design_Client_GUI_Usability_popup}
\end{figure}

The second usability test we conducted used a real implementation of our client, instead of a papermockup. In this test we did not get any feedback that told us that we were lacking user confirmation, which meant that we had solved the issue from test one. The navigation of the client was another matter. The test users found it hard to navigate to the correct pages during the test. They said that they either had to navigate through too many pages ro get to the correct one, or that the buttons were named ambiguously.  For example, one of the users thought that the "View movie" button would play the selected movie.

Unfortunately for us, the second usability test was carried out too late in the process. This meant that we did not have time to incorporate any changes to the client. If we had had the time, we would have done the following: We would have revised the naming of our buttons, and we would make changes to the menu bar at the top of the screen, such that the user would have an easier time navigating through the pages.
