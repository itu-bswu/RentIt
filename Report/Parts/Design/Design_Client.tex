\section{Client}
\label{Design_Client}

\subsection{Analysis}
\label{Design_Client_Analysis}

\subsection{Architecture}
\label{Design_Client_Architecture}

\subsection{Graphical User Interface}
\label{Design_Client_GUI}
When designing the GUI we had two different approachs one where opened a new window each time the user would access a new functionality of the service, and one where we had window in which all the functionallities would be shown. The design team tried out both options and found that having mulitple windows to show the functionallities was to clumsy and would disturb the user more than helping them. Where therefore went with the one window solution since it felt naturally and we could represent new functionallities without disturbing or confusing the user.

\subsubsection{Usability}
\label{Design_Client_GUI_Usability}
When we began designing our GUI, we felt that it was important it was user friendly. Therefore, we choose to make usability tests (see \ref{Testing_Strategy_Usability}), since they always will grant some degree of usability if performed correctly. In total we conducted two usability tests.
\\From the first of the tests, the feedback told us that we didn't have enough user confirmation in the GUI. We discussed this in the design team and came up with a solution, which added dialog and confirmation boxses to a lot of our buttons which contained a save function (see figure \ref{fig:Design_Client_GUI_Usability_popup}on page \pageref{fig:Design_Client_GUI_Usability_popup}).

\begin{figure}[h!]
\caption{Confirmation box}
\label{fig:Design_Client_GUI_Usability_popup}
  \centering
\includegraphics[width=0.75\textwidth]{Parts/Images/Design/Confirmationbox}
\end{figure}

In the second usability test, we went from testing on a papermockup of our client, to testing on our client prototype, which incorporated the design changes from the first test. In this test, we received no feedback concerning the lack of conformation in the client, which meant that we had sovled the problem from test one. We did however get feedback on the navigation of the client. Our test users found it hard to navigate to the correct pages during tests. They said that either they had to navigate through too many pages to get to the correct one or the buttons were named ambiguously. For example, one of the users thought that the view movie button would play the selected movie when clicked.

Unfortunately, the second usability test was carried out too late in the process, which meant that we didn't have time to incorporate changes to the client. If we had had time, we would have done the following: We would have revised the naming of our buttons, such that there would be no confusion with regards to their functionality. In addition, we would makes changes to our "menu bar" at the top of the client, such that the user would allways have more pages to navigate to. This would make unnecessary navigation through other pages redundant.
