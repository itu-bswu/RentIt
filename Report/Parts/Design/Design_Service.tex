\section{Service}
\label{Design_Service}

\subsection{Analysis}
\label{Design_Service_Analysis}

Our service has two layers: the service layer and the logic layer. This is important when we have a public API that other developers can call. We do not want other people direct access to all our logic, but only what we expose as an interface. Our service layer is that interface. It is a wrapper around our logic layer, that validates input before calling an actual method. In our logic class, we may have methods that fails on invalid input, by throwing an exception, or worse, unexpected behavior. That is fine for us to use, as we have scenario tests to ensure that we are not breaking internally. It is a bigger problem when having a public API, since we ca not tell people not to call certain methods in certain ways. We have to validate all input from a client, and that is why we have the service layer on top of our logic layer. The service methods has a standardized way of telling the client, that something went wrong, and will use this, if it detects invalid input. If the input is valid, any error will be a bug in the logic layer, which should not happen if we have enough scenario tests to cover all common use cases. 

The logic layer also handles the connection to the database, which is something that should never be exposed to anyone, but should be part of the private code. Having a security leak, where a 3rd party could get access to our database in any way could be catastrophic, since they could either destroy data to lay down the service, manipulate data for their own interests, or get access to hidden information, such as unreleased movies. That is why the database connectivity should be at least a few layers of abstraction away from anything that is publicly exposed, and that is what we are doing: the database connection itself is handled by Entity Framework, and while we use entity framework directly in the logic layer, we have some abstraction in the layer itself, which means that we mostly does not use it directly anyways. And this layer is still wrapped by the service layer, so that gives a lot of abstraction towards the database, and it should be safe. 

Our tests are also split up into scenario tests for the logic layer, and service tests for the service layer. The scenario tests should check that the logic methods are working, and returning expected output. The scenario tests, on the other hand, does not necessarily test wether or not a service level method returns the correct output - it just checks that a correct input returns any output and no error, while invalid input returns no output and an error. The service level tests should not worry about correct output, since the scenario tests should check this, and we do not want to write all tests twice.

\subsubsection{User Types}
\label{Design_Service_Analysis_UserTypes}
A system user should have a type, which specifies wether the user is just a normal user, or some kind of admin user. A normal user is just someone who has signed up for the service through the signup form, and is able to browse and rent movies. The admin users does not have the ability to rent movies, but are controlling the content in the system. We chose this way to have the same method calls for normal users and admin users, keeping a clean and consistent API. 

There are two kind of admin users: content providers and system admins. The content providers is those who own and manage movies. This could be movie studios, e.g. Universal, or independent movie makers. When logged in as a content provider, you are allowed access to editing and deleting your own movies, as well as registering and uploading new ones. System admins are users, who can manage all content in the system. They can create content providers, and they can view, edit and delete movies. They are also able to ban users, who in some way misuse the system. They ca not rent movies, and they ca not add new movies, but they are a way for the owners of the system to manage content, without having to modify the database directly.

\subsubsection{Movie Editions}
\label{Design_Service_Analysis_MovieEditions}
Typically, movies exists in more editions than just one. A movie can have a directors cut edition, extra material versions, etc., and they can be in both SD, 720p HD and 1080p full HD. Instead of having each version as a new movie in the database, we wanted to group them as editions of the same movie. This is not only to un-clutter movie listings and search results, but also to make it easier for both users and content providers.

We have a special table in the database for editions, and when the user rents a movie, it is actually an edition they rent. On the user interface side, when a user wants to rent a movie, they should be shown all editions of the movie, and be able to choose which one they want. When a content provider is adding a new movie, they first register the movie in the database, without actually uploading any movie data, and they are then able to add editions to the movie. Many movies will have only one edition, but many never movies should be available in both HD and non-HD, which should be uploaded as two different editions. Each edition will have one corresponding movie file uploaded, which is the one the user will receive when renting that edition of the movie.

\subsubsection{Rentals}
\label{Design_Service_Analysis_Rentals}
When a user rents a movie, they will have the ability to download the movie file for as long as the rental period lasts. We decided to have a standard 7 day download period, and if the user wants the movie after that period, he will have to rent it again. There is no payment in the system, but we would have liked to extent the system to make a content provider able to set a price for renting a movie, and possibly also for buying the movie. Other features that could be implemented is discounts on multiple purchases and discount periods.

Having a end date for a rental and allowing content providers to customize prices on movies was some of the ideas suggested by the SMU team, that made it into our feature considerations. 

\subsection{Interface}
\label{Design_Service_Interface}
Our public API should cover 4 different topics: user management (login, logout, sign up), content browsing (get movies, search, get movies in genre, etc.), rental management (rent movie, download movie, view rentals), and content management (register movie, upload edition, edit movie information, delete movie). That's why we have 4 interfaces: IUserManagement, IContentBrowsing, IRentalManagement, and IContentManagement. These four interfaces contains service methods for everything you need.

When designing the API, we had the goals that we wanted it to be simple, easy to figure out, easy to use, small and precise. That's why we decided to have a unified style for all methods: a boolean return value, which is false if an error occurred, a string token as the first parameter, and any return objects as either out or ref parameters. This gives us a unified API, where you don't need to learn how every method works, but are able to use it without much friction.

\subsubsection{Testing}
Other than scenario tests, we have a lot of service level tests. Since the service interface is only a thin wrapper around the actual logic classes, and almost only validates the input, these tests focuses a lot on trying to call the service methods with invalid input. For each method in the service interface, there is one tests that checks that a valid input produces an output and no error. It doesn't necessarily check wether or not the result is correct - that is up to the scenario tests, since the returned value is just the result of a logic-level method call

\subsubsection{SMU involvement}
\label{Design_Service_Interface_SMU}
When collaborating with the SMU team, they pushed to get a interface quick, so the first interface wasn't as well designed as we wanted it to be. It wasn't consistent and it was difficult to extent. An example of this was when we wanted to change the way genres worked: there was no way to do this without breaking the interface. That's why we chose to make the new interface, which improved the old one in every way.

We did exchange ideas with the SMU team about what features should be in the interface, which has been discussed earlier, and some of these has made it even to the new API.

\subsection{Error handling}
\label{Implementation_Service_Error}

We started by throwing exceptions through the service if an error occurred. Down the line we realized that it probably was not a good idea to do it like that. Windows Communication Foundation (WCF) only throws FaultException - even if we throw another exception. The only setting we could tweak was whether or not the original exception was included in the FaultException as an inner exception. This obviously is not easy for the client applications to handle, as they would have to catch FaultExceptions and validate which exception it really was by looking at the inner exception.

The reason is that it is best practice to inform client application of errors in another way. At this time in the project we realized that our system was already built on exceptions. With other, more important tasks to do, we only had time for a simple change.

When we re-designed our service interface (see \ref{Design_Service_Interface}), we used boolean return values for most of the service methods. These would return true for success, and false if any errors occurred (like invalid input). If we were to return anything else we would use \class{out} or \class{ref} parameters.

An even better solution, which we would have implemented if we had the time or had thought about from the start, would be to use enum return types together with the \class{out} or \class{ref} parameters for returning data. The problem with only true/false is that the client application has no way of knowing what went wrong - just that something went wrong. By creating enums we could pass more information about what went wrong to the client application.

For the SignUp service method, the enum could contain the following values:

\begin{my_description}
\item[Success] Signup successful.
\item[UsernameInUse] A user with the specified username already exists.
\item[InvalidEmail] Email specified is invalid.
\item[InvalidPassword] The password specified is invalid (not long enough or empty).
\item[Error] An unknown error occurred.
\end{my_description}

This would provide a lot more information for the client applications, and it would give the client applications to recover and try again.