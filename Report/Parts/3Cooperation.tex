\chapter{Collaboration}
This chapter will focus on how we worked together as a group, how we worked with the SMU team, what problems we ran into and what we could have done differently.

\section{Collaboration}
In the early part of the project, we decided to use the agile development methodology known as ``Scrum''. Scrum is a development model where the team work in ``sprints''. Sprints are a period of time in which the team is supposed to work on set of assignments, determined in the ``sprint planning'' in the beginning of each sprint. These assignments are called "user stories" and they consist of a name, a short description and an estimation of how much time it will take to complete the story. Each user story describes some functionality that the team is supposed to develop. These user stories can be prioritized by the product owner, assuming he or she wants a certain feature finished before another.

Scrum additionally involves daily meetings/stand-up meetings. During these meetings, every team member stands up and take turns to tell the rest of the team what they have been working on since the last meeting, what they intend to work on until next meeting and if something is blocking them. These meetings gives the team a good overview of what is finished, and what is unfinished. Furthermore, it gives the team an opportunity to discuss problems that have been encountered and how to solve these problems.

Another important part of Scrum are the retrospectives. At the end of each sprint the whole team meets, and each team member writes down some points that went well, and some points that did not go well during the sprint. When all members of the team are done, the points are gathered after which they are discussed by the team. After the discussion, the team decides on which points to improve during the next sprint. This allows the team to improve its process and communication over time. This should lead to a better project - both in terms of finished product, but also in terms of how the team works together.

\section{ITU group structure}
\label{Collaboration_ITU}
Having decided to use Scrum for our work, we had to distribute the roles that we felt were important to our work. This meant that we had the following four roles to distribute:
\begin{my_description}
\item[Product Owner]
The Product Owner is the person/company who has ordered the product. Normally this would be a specific person or company, but in our case that is not entirely true. Technically, our lector is the one who has "ordered" the product, which normally would make him the product owner. However, we also had to take the Singaporeans' opinions into consideration. Considering that they should use the product as well, they also had a say in how the product was supposed to work.

In the end, we decided that the Product Owner should be a combination of our lector and our Team Leader. The reason behind this is that they are the ones who decides whether the product meets the expectations and requirements or not.

\item[Team Leader]
The Team Leader has some responsibilities. His job is to make sure that the team finishes their tasks on time, that the team is communicating, that the team is functioning well, etc. The Team Leader is basically responsible for the whole team. If the Product Owner has a problem with regards to the team, he should contact the Team Leader and let the Team Leader take care of it. We chose Niklas Hansen for this role.

\item[Scrum Master]
The Scrum Master is responsible for ensuring that the Scrum process is used as intended. A key part of his role is to keep the team focused on the user stories and make sure that the team is not distracted by outside influences. In our group we also decided to let the Scrum Master control the daily meetings, the sprint plannings and the retrospectives. We chose Frederik Lysgaard for this role.

\item[QA Responsible]
The QA Responsible is responsible for looking at stories that have been finished, and making sure they work as intended. The QA responsible is the one who decides if a user story has been finished or not. This is done by running tests that have been written for the story and making sure that the tests covers the user story properly. We chose Jakob Melnyk for this role.
\end{my_description}

We had to decide whether we wanted the roles to rotate among the team members, or let people keep their roles until the end of the project. After some discussion, we decided to keep the roles static. Had we changed them ever so often, it would cause confusion for both us and the SMU team.

\subsection{Meetings}
\label{Collaboration_ITU_Meetings}
As mentioned in the description of Scrum, meetings are an important part of Scrum. This meant that we had a lot of focus on meetings, which will be discussed in this section.

In the beginning we decided to meet on Mondays from 12.00 to 14.00, along with Tuesdays and Thursdays from 12.00 to 16.00. After finishing our other courses, we decided to meet every day, except Friday and Sunday, from 10.00-16.00. We would start the day with a stand-up meeting. During the meeting, we told each other what we had been working on since the last meeting, and what we intended to work on until the next meeting. After the stand-up meeting, we would begin working on our tasks.

Stand-up meetings were not the only kind of meeting we had. Monday before a sprint ended, we would hold a retrospective meeting, in which we would discuss how the sprint had gone, and what we should improve on. With these meetings we kept improving our work and our communication within the team.

The last type of meetings we used, were the sprint planning meetings. During these meetings we would look at new stories, estimate them and prioritize them, according to which stories our product owner wanted to have finished first. We estimated them using planning poker, where each team member estimated how long a story would take to complete. The ones with the lowest and the highest estimate would explain their estimate, and with that in mind, we would do planning poker again until we all agreed. We would then assign the stories to team members, after which we would be working on them. The sprint planning meetings were also used to look at our status, see how many stories we had completed in the previous sprint, and how many that were left. The ones left from previous sprints would be prioritized higher in the next sprint, so we could finish them and begin working on new ones.

\section{SMU cooperation}
\label{Collaboration_SMU}
Working in teams on the ITU side was not the only form of teamwork in this project. It was planned that we should collaborate with a group from the Singapore Management University during the project. This meant that another aspect was added to the project, as we were to communicate with people whom we had never met. It turned out to be more of a challenge than expected, as we learned during the course.

We were introduced to the SMU team on the 6th of March. During this meeting, we agreed to use Google+ Hangout (a video conference tool) as our method for communicating during meetings. For communication that did not relate to the meetings, we agreed on using email, as we feel it is an efficient tool for communication. In addition it has the advantage that everything that is sent back and forth is documented, and thus can be looked up at a later date.

\subsection{Meetings}
\label{Collaboration_SMU_Meetings}
Our meetings with the SMU team were scheduled to be every Thursday around 13.00-15.00. We had a total of 5 meetings with the SMU team, in which we would update each other on how we were doing. After the update we would talk about what we were planning on working on until the next meeting. The meetings were additionally used for sharing ideas about what both groups wanted the service to be able to do.

\subsection{Conflicts}
\label{Collaboration_SMU_Conflicts}
Working with the SMU team was quite a new experience for us, as no one from our team had tried working with another group of people on the other side of the globe before. The only expectations we had, were from what we had been told during the lectures. Therefore, we were quite nervous going into the project. But the first video conference with the SMU team went really well, and gave real hope of a successful collaboration. But as time went by, we experienced different problems, which will be discussed here.

\subsubsection{Mood changes}
\label{Collaboration_SMU_ConflictsMood}
The use of Google+ Hangout as a video conference tool improved our communication during the meetings. Unfortunately, it did not mean that we were able to predict some of the ``mood changes'' that the Singaporeans had. During our meetings we would come to an agreement on a certain subject, and the next day or so we would receive an email asking whether we could do it another way instead. Even though these suggestions typically were quite a bit different than what we decided, we would consider the pros and cons and come up with a compromise.

\subsubsection{Wrong API}
\label{Collaboration_SMU_ConflictsAPI}
At some point in time we received an email with questions about our service API. We looked over their feedback, and was slightly confused, as we could not recognized any of the service methods they were mentioning in their email. After a brief email conversation we found out that they had been looking at another group's API, which did not match ours. We emailed them our API again, and had them look it over.

\subsubsection{Behind schedule}
\label{Cooperation_schedule}
We also had some problems on our side, which impacted their side. For example we had some database issues at the beginning of the project, which prevented us from doing much for a week. The database issue put us behind schedule, and because of that, the Singaporeans were put behind schedule.

\subsubsection{Error reporting}
\label{Cooperation_error}
The fact that the Singaporeans were put behind schedule turned out to be a major problem, as they were rather slow to report when they encountered problems. Towards the end of their schedule, they were unable to make our service work for them, and they did not tell us until they were approaching their own deadline. We managed to solve the issue, though it could have been handled a lot better and faster if we had received their report earlier.

At the start of the project, we regularly received emails with what issues they encountered, so when they stopped reporting any issues, we assumed everything was fine. But as mentioned above, this was not the case.

\subsection{What we could have done differently}
\label{Collaboration_SMU_CouldHave}
The problems mentioned in section \ref{Collaboration_SMU_Conflicts} happened because of one major problem: Bad communication. Not only did we not have enough communication between the teams, but our communication was not very clear. Because our communication lacked clarity at times, we had a few cases of complete misunderstandings, which took a lot of extra communication to figure out.

As stated at the beginning of this chapter, we used Scrum on the ITU side. The SMU team told us during one of our weekly video conferences they were using Scrum, and they would adapt to our Scrum process and sprints. However, during the process, we did not see much evidence of an incremental approach to development. They were expected a fully functional service right away, where we were working on it in incremental steps.

If we had spent some more time talking with the SMU team at the beginning of the project, we would have been able to figure out how exactly they were running their part of the project, and we could have explained how what development model we were using. This way we would know what to expect from each other, and we would have an easier time figuring out how to help each other when needed.

While clear and verbose communication is, in our opinion, to be preferred, more communication could also have helped resolve the issues with bad communication. The group as a whole did not have communicate that much with the SMU team. It was mainly our Team Leader (Niklas Hansen) and our QA Responsible (Jakob Melnyk) who took care of the communication. This meant that some things were lost when other team members had some input on a subject.

We should, over the whole course, have been more adamant about receiving regular updates from them. Sometimes we would hear nothing from them and when they actually replied to status updates, we did not get much insight into what their status actually was. Had we been more adamant about receiving updates, we could have handled a lot of problems earlier, easier and faster. But that is an experience we can take with us for our next project.