\chapter{Collaboration}
This chapter will focus on how we worked together as a group, how we worked with the SMU team, what problems we ran into and what we could have done differently.

\section{Collaboration}
When we began the project, we decided to use the agile form for development known as "SCRUM". SCRUM is a development form where the team works in "sprints". Sprints are a period of time in which the team is supposed to work on certain features. These features are called "user stories" and they consist of a name, a short description and an estimation of how much time it will take to complete the story. Each user story descirbes a feature that the team is supposed to develop. These user stories can be prioritised by the product owner, assuming that the product owner wants a certain feature finished before another.

SCRUM also contains daily meetings/stand-up meetings. In these meetings, every team member stands up and, one after the other, tells the rest of the team what they have been working on since the last meeting, what they intend to work on until next meeting and if anything can prevent them from getting it done. These meetings gives the team a good overview of what has been finished, and what is left to finish. Furthermore, it gives the team an opportunity to discuss problems that have been encountered, and how to solve these problems.

Another important part of SCRUM are the retrospectives. At the end of every sprint, the whole team meets, and each teammember writes some points that went well, and some points that did not go well during the spring. These points are all gathered, after which they are discussed by the team. After the discussion, the team decides on which points to improve during the next sprint. This allows the team to improve over time. This should lead to a better project, both in terms of finished product, but also in terms of how the team works together.

\section{ITU group structure}
\label{Collaboration_ITU}
Having decided upon using SCRUM for our project, we had to distribute the roles that we felt were important to our work. This meant that we had the following 4 roles to distribute:
\begin{my_description}
\item[Product Owner]
The Product Owner is the person/company who has ordered the product. Normally, this would be a specific person or company, but in our case that isn't entirely true. Technically, our lector is the one who has "ordered" the product, which normally would make him the product owner. However, we also had the Singaporeans to take into account. Considering that they should use the product as well, they also had a say in how the product was supposed to work.

In the end, we decided that the Product Owner should be a combination of our lector and our Team Leader. The reason behind this is that they are the ones who decides whether the product lives up to the demands or not.

\item[Team Leader]
The Team Leader has some responsebilities. His job is to make sure that the team finishes their tasks on time, make sure that the team shows up on the agreed time, make sure that the team is functioning well, etc. The Team Leader is basically responsible for the whole team. If the Product Owner has a problem with regards to the team, he should contact the Team Leader and let the Team Leader take care of it.

In our team, we had two candidates for the Team Leader role; Niklas Hansen and Jakob Melnyk. After some talking, Jakob Melnyk decided to let Niklas Hansen get the Team Leader role, as it turned out that it was not something he was very interested in.

\item[SCRUM Master]
The SCRUM Master is responsible for ensuring that the SCRUM process is used as intended. A key part of his role is to keep the team focused on the user stories, and make sure that the team is not distracted by outside influences. In our group, we also decided to let the SCRUM Master control the daily meetings, the sprint plannings and the retrospectives.

For our SCRUM Master role, we decided upon Frederik Lysgaard, as he was the one who showed the most interest in the role.

\item[QA Responsible]
The QA Responsible is responsible for looking at stories that have been finished, and making sure that they work as intended. The QA responsible is the one who decides if a user story has been finished or not. This is done by running tests that have been written for the story and making sure that the tests covers the user story well enough.

The QA role was given to Jakob Melnyk because he is very good at making sure that things work as are they are supposed to.
\end{my_description}

The second thing we had to do with regards to roles, was to decide whether we wanted the roles to rotate on a certain basis, or let people keep their roles until the end of the project. After some discussion, we decided to keep the roles static. Had we changed them every so often, it would cause confusion for both us and the SMU team, which is why we decided to keep the roles static.

\subsection{Meetings}
\label{Collaboration_ITU_Meetings}
As mentioned in the descruption of SCRUM, meetings are an important part of SCRUM. This meant that we had a lot of focus on meetings, which will be discussed in this section.

In the beginning, we decided to meet on Mondays from 12.00 to 14.00, along with Tuesdays and Thursdays from 12.00 to 16.00. After finishing our other subjects, we decided to meet every day, except Friday and Sunday, from 10.00-16.00. When we met, we would start off with a stand-up meeting. during the meeting, we told each other what we had been working on since the last meeting, and what we intended to work on until the next meeting. After the stand-up meeting, we would begin working on our tasks.

Stand-up meetings weren't the only kind of meeting we had. Every second Tuesday we would hold a retrospective meeting, in which we would discuss how the sprint had gone, and what we should improve on. Using these meetings, we kept improving our work, both with regards to how we worked, but also with regards to the quality of our product.

The last type of meetings we used, were the sprint planning meetings. On these meetings we would look at new stories, estimate them and prioritise them, according to which stories our product owner wanted to have finished first. We would then assign the stories to team members, after which we would working on them. The sprint planning meetings were also used to look at our status, see how many stories we had completed in the previous sprint, and how many that were left. The ones left from previous sprints would be prioritised higher in the next sprint, so we could have them finished and begin working on new ones.

\section{SMU cooperation}
\label{Collaboration_SMU}
Working in teams on the ITU side wasn't the only form for teamwork that was to be done in this project. It was planned that we should collaborate with a group from the Singapore Management University during the project. This meant that another ``dimension`` was added to the project, as we suddenly were to communicate with people whom we had never met, and whose skills we knew nothing about. It turned out to be more of a challenge than expected, as we learned during the course.

We were introduced to the SMU team on the 6th of March. During this meeting, we agreed on using Google+ Hangout (a video conference tool) as our method for communicating during meetings. For communication that did not relate to the meetings, we agreed on using email, as it is an efficient tool for communication. It also has the advantage that everything that is sent back and forth is documented, and thus can be looked at at a later date.

\subsection{Meetings}
\label{Collaboration_SMU_Meetings}
Our meetings with the SMU team were scheduled to be every Thursday around 13.00-15.00. We had a total of 5 meetings with the SMU team, in which we would update each other on how we were doing. After the update, we would talk about what our plans were until the next meeting. The meetings were also used for sharing ideas about what both groups wanted the service to be able to do.

\subsection{Conflicts}
\label{Collaboration_SMU_Conflicts}
Working with the SMU team was quite a new experience for us, as no one from our team had worked with a team from that far away before. The only expetations we had, were from what we had been told during the lectures. Therefore, we had hoped that working with the SMU team would be relatively painless, and it looked like that was the case at the beginning. But as time went by, we ran into different problems, which will be discussed here.

\subsubsection{Mood Changes}
\label{Collaboration_SMU_ConflictsMood}
The use of Google+ Hangout as a video conference tool improved our communication during the meetings. Unfortunately, it did not mean that we were able to predict some of the ``mood changes'' that the Singaporeans had. During our meetings we would agreed with them on something, and the next day we would receive a mail saying ``can we do it this way instead? '', with their suggestion usually being something completely opposite of what we had agreed on. Whenever this happened, we would end up having an email conversation with the Singaporeans, and in the end we would come to a solution that both sides would agree on.

\subsubsection{Wrong API}
\label{Collaboration_SMU_ConflictsAPI}
Another problem we encountered, was that they had spent some time looking at an API, before they sent us an email with questions about it. When we received the questions, we had no idea what they were talking about. Somehow, they had managed to find an API that wasn't ours, and they had been looking at that one instead of ours. This lead to confusion and a mail conversation, but in the end we managed make them look at the correct API.

\subsubsection{Misconception}
\label{Collaboration_SMU_ConflictsMisconception}
This is not to say that they created all the problems. On our end there was some misconception with regards to what they were capable of, when it came to programming. We thought that they were about our level, but it turned out that they weren't. This meant that some things were not done correctly on their end, which resulted in extra work for both ends.

\subsubsection{Behind schedule}
\label{Cooperation_schedule}
We also had some problems on our side, which impacted their side. For example we had some database issues at the beginning of the project, which prevented us from dooing much for a week. The database issue put us behind schedule, and because of that, the Singaporeans were put behind schedule.

\subsubsection{Error reporting}
\label{Cooperation_error}
The fact that the Singaporeans were put behind schedule turned out to be a major problem, as they were rather slow to report when they encountered problems. Towards the end of their schedule, they were unable to make our service work for them, and they didn't tell us until they were approaching their own deadline. We managed to solve the issue, though it could have been handled a lot better and faster if we had received their report earlier.

\subsection{What we could have done differently}
\label{Collaboration_SMU_CouldHave}
The problems mentioned in section \ref{Collaboration_SMU_Conflicts} can be said to have happened because of one problem: Bad communication. Not only did we not have enough communication between the team teams, but our communication was not very clear. Because our communication lacked clarity at times, we had a few cases of complete misunderstandings, which took a lot of extra communication to figure out.

As stated at the beginning of this chapter, we used Scrum on the ITU side, and from our second meeting with the SMU team, we thought they were using Scrum as well. However, during the process, it felt like they were using another form for Scrum than we were, if they were even using Scrum at all. They wanted to implement the full service at once, instead of working on it in incremental steps.

If we had spent some more time talking with the SMU team at the beginning of the project, we would have been able to figure out how exactly they were running their part of the project, and we could have explained how what development model we were using. This way we would know what to expect from each other, and we would have an easier time figuring out how to help each other when needed.

While clear and verbose communication is, in our opinion, to be preferred, more communication could also have helped resolve the issues with bad communication. The group as a whole did not have communicate that much with the SMU team. It was mainly our Team Leader (Niklas Hansen) and our QA Responsible (Jakob Melnyk) who took care of the communication. This meant that some things were lost when other team members had some input on a subject, due to human errors.

We should, over the whole course, have been more adamant about receiving regular updates from them. Sometimes we would hear nothing from them and when they actually replied to status updates, we did not get much insight into what their status actually was. Had we been more adamant about receiving updates, we could have handeled a lot of problems earlier, easier and faster. But that is an experience we can take with us for our next project.