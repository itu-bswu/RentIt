\chapter{Collaboration}
\label{Collaboration}
This chapter will focus on how we worked together as a group, how we worked with the SMU team, what problems we ran into and what could have been done differently.

\section{Development model}
\label{Collaboration_Development}
When we began the project, we decided on using the agile form for development known as "Scrum". Scrum is a development form where the team works in "sprints", which are a period of time in which the team is supposed to work on certain features. These features are called "user stories", which consist of a name, a short description and an estimation of how much time it takes to complete the story. Each user story describe a feature that the team is supposed to develop over the current sprint, and they can be prioritised by the product owner if the product owner wants a certain feature finished before another feature.

Scrum also contains daily meetings/stand-up meetings, in which everyone from the team stands up and tells what they have been working on since the last meeting, what they intend to work on until next meeting and if anything can prevent them from doing this work. These meetings gives the team a good overview of what's been finished, and what still needs finishing. Furthermore, it gives the team an opportunity to discuss problems that have been encountered, and how to solve these problems.

Another important part of Scrum is the so-called "retrospective". At the end of every sprint, each teammember writes down some good and bad things that happened during the sprint. These points are all gathered, and discussed by the team, after which the team decides on what points to improve during the next sprint. This allows the team to improve over time. This should lead to a better project, both in terms of the finished product and in how the team works together.

\subsection{ITU group structure}
\label{Collaboration_Development_ITU}
Having decided upon using Scrum for our project, we had to distribute the roles we wanted to use. More precisely, we had the following 4 roles to distribute: 
\begin{my_description}
\item[Product Owner]
The Product Owner is the person/company who has ordered the product. Normally, this would be a specific person or company, but in our case that isn't entirely true. Techinically, our lector is the one who has "ordered" the product, which normally would make him the product owner. However, we also had the Singaporeans to take into account, as they had a say in how the product should function, as they needed to be able to use it as well.

In the end, we decided that the Product Owner should be a combination of our lector along with our Team Leader, as they are the ones who confirm whether the product meets the requirements.

\item[Team Leader]
The Team Leader is responsible for making sure that people get their stuff done on time, making sure that people show up to the agreed time, making sure that the team is functioning well etc. The Team Leader is basically responsible for the whole team. If the Product Owner has a problem with regards to the team, he should contact the Team Leader and let the Team Leader take care of it.

In our team we had two candidates for the Team Leader; Niklas Hansen and Jakob Melnyk. After some talking, Jakob decided to let Niklas tak on the role of Team Leader because Jakob did not want to be the leader of the group.

\item[Scrum Master]
The Scrum Master is responsible for making sure the Scrum process is used as intended. A key part of his role is to keep the team focused on the user stories, and to make sure that the team is not distracted by outside influences. In our group, we decided to let the Scrum Master take control of the daily meetings, the sprint plannings and the retrospectives.

We chose Frederik Lysgaard as our Scrum Master as he showed interest in the role.

\item[QA Responsible]
The QA Responsible is responsible for taking a look at stories that have been finished, and making sure that they work as intended. This includes running tests that have been written for the story by the teammember who was responsible for the development of that specific story.
The QA role was given to Jakob Melnyk because he is very good at making sure that things work as are they are supposed to.
\end{my_description}

The second thing we had to do with regards to roles, was to decide whether we wanted the roles to rotate on a certain basis, or just let people keep the roles until the end of the project. After some discussion, we decided to keep the roles static. Had we changed them from time to time, it would cause confusion for both us and the SMU group, which is why we decided to keep the roles static.

\subsection{Meetings}
\label{Collaboration_ITU_Meetings}
As mentioned in the descruption of Scrum, meetings are an important part of Scrum. This meant that we had a lot of focus on meetings, which we discuss in this section.

In the beginning we decided to meet on Mondays from 12.00 to 14.00, and Tuesdays and Thursdays from 12.00 to 16.00. Later on we decided to meet every day except Friday and Sunday, from 10.00-16.00. When we met during these times, we would begin with a stand-up meeting. At the stand-up meetings we told each other what we had been working on since the last meeting and what we intended to work on until the next meeting. After the stand-up meeting we would begin working on our tasks.

Stand-up meetings weren't the only kind of meeting we had, though. Every second Tuesday we would have our retrospective meeting, in which we would discuss how the sprint had gone, and what we could improve. Using the discussions we had at the retrospectives we improved the way we work together over the course of the project.

The last type of meetings we had were the sprint planning meetings. At these meetings we would take a look at new stories, estimate them and prioritise them, according to which stories our product owner wanted finished first. We would then assign the stories to team members for them to work on. The sprint planning meetings were also used to take a look at our status, see how many stories we had completed in the previous sprint, and how many that were left. The ones left would be prioritised higher in the next sprint, so we could have them finished and begin working on new ones.

\section{SMU Collaboration}
\label{Collaboration_SMU}
We were to work in teams on the ITU side, but that wasn't the only teamwork that was to be done in this project. It was planned that we should collaborate with a group from Singapore Management University during the project. This meant that another ``dimension`` was added to the project, as we suddenly were to communicate with people whom we had never met, and whose skills we knew nothing about. It turned out to be more of a challenge than expected, as we learned during the course.

We were introduced to the SMU team the 6th of March, where we agreed on using Google+ Hangout (video conference tool) as our method for communicating during meetings. For communication that did not relate to the meetings, we agreed on using email, as it's an efficient tool for communication. It also has the advantage that everything that is sent back and forth is documented, and thus can be looked at at a later date, if need be.

\subsection{Meetings}
\label{Collaboration_SMU_Meetings}
The meetings with SMU were scheduled to be every Thursday around 13.00-15.00.We had a total of 5 meetings with the SMU team\footnote{Meeting logs are included in \ref{Appendix_SMU} on page \pageref{Appendix_SMU}.}. At the these meetings we would discuss our progress on both ends and what we were planning to do until the next meeting.. The meetings were also used for sharing ideas about what both groups wanted the service to be able to do.

\subsection{Conflicts}
\label{Collaboration_SMU_Conflicts}
Working with the SMU team was quite a new experience for us, as no one from our team had worked with a team from that far away before. The only expectations we had were the one we got from Niels\footnote{16th of February, Niels Hallenberg lecture on Inter Cultural communication.} With this in mind we were quite surprised when things seemed to be going very well in the early parts of the project. But as time went by, we ran into different problems with regards to working with the SMU team, which we discuss in the following sections.

\subsubsection{Mood Changes}
\label{Collaboration_SMU_Conflicts_Mood}
The use of Google+ Hangout as a video conference tool improved our communication during the meetings. Unfortunately, it did not mean that we were able to predict some of the "mood changes'' that the Singaporeans had. During our meetings we'd agreed with them on something, and the next day we'd receive an email saying "Can we do it this way instead?", with their suggestion usually being something completely opposite of what we had agreed on. Whenever this happened, we'd end up having an email conversation with the Singaporeans, and in the end we would come to a solution that both sides would agree on.

\subsubsection{Wrong API}
\label{Collaboration_SMU_Conflicts_API}
This wasn't the only problem we encountered with regards to communication. They had spent some time looking an API over before they sent us an email with questions about it, and when we received the questions we had no idea what they were talking about. Somehow, they had managed to find an API that wasn't ours, and they had been looking at that one instead of ours. This lead to confusion and a mail conversation, but in the end we managed to get them to look at the correct API.

\subsubsection{Misconception}
\label{Collaboration_SMU_Conflicts_Misconception}
On our end there was also some misconception with regards to what they were capable of with regards to system development. We thought that they were about our level when it came to programming, but it turned out that they were not. This meant that some things were not done correctly on their end, which resulted in extra work and a lot of emails between the groups.

\subsubsection{Behind schedule}
\label{Collaboration_SMU_Conflicts_Schedule}
This is not to say that they created all the problems, as we had some problems on our side as well, problems which impacted their side. For example we had some database issues in the beginning of the project, which pretty much prevented any development from us for a week. The database issue put us behind schedule, and because of that the Singaporeans got behind schedule.

\subsubsection{Error reporting}
\label{Collaboration_SMU_Conflicts_error}
The fact that the Singaporeans were put behind schedule turned out to be a major problem, as they were rather slow to report when they encountered problems. Towards the end of their schedule, they were unable to make our service work for them, and they didn't tell us until they were approaching their own deadline. We managed to solve the issue, though it could have been handled a lot better and faster if we had received their report earlier.

\subsection{What we could have done differently}
\label{Collaboration_SMU_CouldHave}
The problems mentioned in section \ref{Collaboration_SMU_Conflicts} can be said to have happened because of one problem: Bad communication. Not only did we not have enough communication between the team teams, but our communication was not very clear. Because our communication lacked clarity at times, we had a few cases of complete misunderstandings, which took a lot of extra communication to figure out.

As stated at the beginning of this chapter, we use Scrum on the ITU side, and from our second meeting with the SMU team, we thought  they were using Scrum as well. However, during the process, it felt like they were using another form for Scrum than we were, if they were even using Scrum at all, as they wanted to implement the full service at once, instead of working on it in incremental steps.

If we had spent some more time talking with the SMU team at the beginning of the project, we would have been able to figure out how exactly they were running their part of the project, and we could have explained how what development model we were using. This way we would know what to expect from each other, and we would have an easier time figuring out how to help each other when needed.

While clear and verbose communication is, in our opinion, to be preferred, more communication could also have helped resolve the issues with bad communication. The group as a whole did not have the 

Another thing we could have done as well, was simply to communicate more. We didn't have that much communication with the SMU team overall. The largest part of our communication happened during the meetings, and over the email conversations after each meeting, where questions were asked and answered. Between the meetings, the amount of conversation was fairly small, which of course meant that we didn't know how the SMU team was doing.

We should probably have been more adamant about receiving regular updates from them. Sometimes we would hear nothing from them and even when they actually replied to status updates, we did not get much insight into what their status actually was.

The last thing we could have done differently, was to be more insistent when it came to getting updates from them. We sent some emails to them once in a while, asking for status on their end, without recieving an answer. This lead us to believe that everything was fine on their end, which it turned out that it wasn't.	

In short, more, and better, communication from the beginning would have done wonders for the project as a whole.