\section{Client}
\label{Implementation_Client}
This section describes the implementation of our client, the issues/bugs in the code, the fixes and workarounds we used to circumvent those issues and how we handled errors.
\subsection{Architecture implementation}
\label{Implementation_Client_Architecture}
Our implementation of the client is seperated into two parts. The implementation of the \class{MVVM} architectural pattern and the implementation of the \class{WPF} framework.
\subsubsection{Model-View-ViewModel}
\label{Implementation_Client_Architecture_MVVM}
Our intention is for the client to follow a simple dependency flow such as described by figure \ref{fig:Implementation_Client_Architecture_MVVM_Namespace} on page \pageref{fig:Implementation_Client_Architecture_MVVM_Namespace}. The \class{View} (\class{GUI}) should only request information from the \class{ViewMode}l and should never be called by anything but classes from the GUI namespace. Similarly the \class{ViewModel} should only be called by their respective \class{View} classes\footnote{This is described in more detail in \ref{Appendix_Diagrams_Class_Client} on page \pageref{Appendix_Diagrams_Class_Client}.}. 
\begin{figure}[h!]
  \centering
    \includegraphics[width=0.5\textwidth]{Parts/Images/Implementation/NamespaceDependency}
  \caption{Namespace dependency in the client}
\label{fig:Implementation_Client_Architecture_MVVM_Namespace}
\end{figure}

At the "bottom" level of our architecture, the \class{Model} classes should only be called by eachother and the \class{Viewmodel}s (although no \class{Viewmodel} should ever call the \class{ServiceClient} class). The only calls outside the namespace that the \class{Model} should make is to the RentItService (not included in figure \ref{fig:Implementation_Client_Architecture_MVVM_Namespace}). The actual data source that the \class{Model} classes use should not be of consequence to the \class{ViewModel} or the \class{GUI}. 

\paragraph{Static classes}
The reason we use static classes and methods for the \class{ViewModel} and \class{Model} is due to the actual application being a singleton in itself, but the way we use \class{WPF}\footnote{See section \ref{Implementation_Client_Architecture_WPF} for our implementation of the \class{GUI} using \class{WPF}.} means that we do not have access to static objects in the \class{GUI}.

\paragraph{Upload/download} Our upload and download implementation is heavily inspired by "Streaming upload/download files over HTTP" on \cite{CODE:UPLDL}. 

\subsubsection{Windows Presentation Foundation}
\label{Implementation_Client_Architecture_WPF}
Our implementation of the \class{View} is done in \class{WPF}. It involves a single \class{MainWindow} which is essentially a \class{WCF} \class{NavigationWindow}. It opens up a \method{LoginPage} inside the window. Whenever we change to a new page (for example on login), we use the \method{NavigationService} of the window to open a new page.
\\The way we navigate using the navigation service means we have to create new \method{Page} objects every time we want to change to a new page.

\subsubsection{Issues, workarounds and fixes}
\label{Implementation_Client_Architecture_Issues}

\subsection{Error handling}
\label{Implementation_Client_Error}

\class{}